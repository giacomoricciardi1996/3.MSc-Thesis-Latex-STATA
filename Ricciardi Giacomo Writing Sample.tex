\documentclass[12pt]{article}
\renewcommand{\baselinestretch}{1.5}
\usepackage{hyperref}
\usepackage{apacite}
\usepackage[english]{babel}
\usepackage[margin=2.5cm]{geometry}
\usepackage{amsmath}
\usepackage{color,soul}
\usepackage{graphicx}
\usepackage[gen]{eurosym}
\usepackage{rotating}
\graphicspath{ {./Graphs/} }
\usepackage{booktabs}
\usepackage{float}
\usepackage{appendix}
\usepackage[table,xcdraw]{xcolor}
%\restylefloat{table}
\usepackage[font=scriptsize,labelfont=bf,justification=justified,format=plain]{caption}
\captionsetup{skip=0pt}
\setlength\belowcaptionskip{-3ex}
\captionsetup[figure]{justification=raggedright,singlelinecheck=false}
\captionsetup[table]{singlelinecheck=false}

\hypersetup{
    colorlinks, %activate in case you want the border to the citations
    citecolor=black,
    filecolor=black,
    linkcolor=black,
    urlcolor=black
}
\interfootnotelinepenalty=10000
%
%\usepackage{tikz} % To generate the plot from csv
%\usepackage{pgfplots}
%\pgfplotsset{compat=1.5}
%


\begin{document}

\title{Factor-based Investing: \\ empirical analysis on style factors during different types of crisis}
\author{Giacomo Ricciardi}

\date{}
\maketitle{}

\begin{abstract}
Factor-Based investing has become an extensively discussed subject in today’s asset management universe. In particular, by Factor-Based investing one means a set of investment strategies focusing on specific securities characteristics that are fundamental in explaining their risk and return. The aim of this work is to investigate how do risk factors behave during six of the most critical crises of the last thirty years, namely the Asian Crisis of 1997, the 1998 turmoil, the Dot-com Bubble, the 2008 Financial Crisis, the peak of the European Debt Crisis and, most importantly, the COVID-19 Pandemic. 

Firstly, factors are built by implementing self-financing strategies, i.e. going long one dimension by shorting the opposite exposure. The two datasets (one world-based and the other US-based) are then subdivided in order to consider the aforementioned crises by also maintaining a robust number of observations.

The methodology part is then split in two complementary building blocks. The first summarizes the risk-return performances of equity styles (market, valuation, size, momentum, low volatility and quality), fixed income risk premia (credit, high yield and term spread) and the typical asset classes, despite the emphasis is mostly put on equity-oriented styles throughout the whole work. Key findings over the whole period (1997-2020) and by single crisis are discussed, with a focus on COVID-19, where the evidence supports the premise of factor investing not being excluded from the aftermaths of the pandemic. The second subsection implements a crisis regime analysis, whose purpose is to understand how the returns of the worst performing stocks were explained by the different factors in the different crises. Hence, several time-series regressions are run to highlight how were these firms exposed to equity styles, before entering the debacles.

Overall, style factors behave in a heterogenous manner when considering these six crises, but some common elements are often verified. Conclusions are coherent with existing literature, with particular evidence of the importance of crises’ nature, but with encountered cross-crisis patterns for valuation and size, as these two suffered the most regardless of the intrinsic essence of the debacles themselves. The severity of COVID-19 on factor investing is also a crucial outcome, in addition to the absence of evidence of a definitive winner between factor-based investing vs. asset class investing. Re-running the analysis by building the long- short portfolios thanks to exposure-stock-picking represents the most intriguing follow-up research, despite expected discrepancies with the usage of indices, i.e. as done in this work, remain limited.
\end{abstract}

\newpage

\tableofcontents
\setcounter{page}{1}

% Introduction paragraph: a paragraph describing: the topic of the study, its positioning within the relevant literature, highlighting a summary of the “gaps” in the knowledge that the research thesis aims to fill, together with the research questions that gave rise to the study, the contributions offered by the research thesis, a brief illustration of its structure.

\newpage

\newpage
\section{Introduction} \label{section:1}

\subsection{Factor-Based investing}
Factor-Based investing has become an extensively discussed subject in today's asset management universe. In particular, by Factor-Based investing one means a set of investment strategies focusing on specific securities characteristics that are fundamental in explaining their risk and return. Hence, it provides a framework that allows to incorporate factor-exposure choices into the daily life of portfolio managers. Numbers and projections fully sustain the importance of this topic: according to BlackRock, the factor industry is expected to grow to \$3.4 trillion by 2022 \cite{what_is_factor_investing}, with the factor ETF assets (i.e. factor investing done through ETF) already accounting for \$482 billion as of October 31, 2019 \cite{factor_capacity}.

However, what do people in the industry aim to achieve with factor-investing? In a few words, how did the usage of this tool become so diffused in the last years? The answer is not unique, but it can find its roots in the structure of financial markets themselves. Indeed, these are composed of several asset classes and securities, leading to major difficulties when it comes to understanding what really drives the performance of a portfolio and why.

Nonetheless, some of these drivers are well recognized, and they are taken into account whether the objective of the fund is diversification, risk-return performance targets or a tailor made strategy. It is well known, for example, that there is a tendency for winning stocks to keep performing well both in the small-cap universe and in the large-cap one. Properly talking about size, undertaking long positions in stocks with small market capitalization while shorting large market cap companies has been proven to be a strategy leading to positive returns over time. These were just two examples to give a general idea of the fact that investors' appetite for outstanding performance has fueled the understanding of the performance itself, hence leading to new investment strategies, such as factor-investing.

Even though researchers have historically put their emphasis on stocks-related factors, these are not an exclusive variable of the equity world. The world of fixed income, as an instance, has been influenced by these theories as well, with new studies trying to understand how bearing exposure in this sphere is expected to pay out risk premia. However, when abstracting from the market they relate to and only considering their original nature, it could be possible to subdivide factors into two broad categories: macroeconomic (i.e. macro) and style factors \cite{bender2018asset}. The former refer to all kind of drivers influencing returns across asset classes. The latter, instead, represent all the treats of securities and asset classes that are helpful to explain excess returns. While macro factors are intuitive and straightforward to understand, research has increasingly shifted towards style factors \cite{invesco_factor_investing}. 

Both investing in macro and / or style factors can be more or less profitable than doing it in asset classes directly and, of course, more or less exposed to the threats our economy suffers and the plunges that financial markets bear during distress periods. Despite the broad topic is well documented in past literature on a general level, the behavior of factor-investing in the major crises of the last 30 years has not been vastly investigated yet. Being more precise, it is not uncommon for the milestone papers tackling macro and style factors to discuss the performance of these investment strategies during \textit{one} financial pressure, but not so common to find one that summarizes results considering \textit{more of them at the same time}. Furthermore, the recent pandemic the world is experiencing represents an unexplored land when it comes to factor-investing. 

\subsection{The COVID-19 crisis}
Hence, before diving into the research question of this work by digging into factor-investing in details, it is worth discussing the  COVID-19 pandemic situation we are experiencing, with one eye firmly kept on the macroeconomic reality and the other focused on financial markets. 

If we concentrate on the former aspect, one of the crucial component to be considered is debt. Governments have dramatically increased their debt-to-GDP ratios to sustain the crisis. Hoping for going back to normality as soon as possible, citizens will demand more and more in public services in the next months, leading to the twofold consequences of being more exposed to a second lockdown and nourishing those already-pumped leverage ratios. Furthermore, with the US elections behind the corner, fulfilling people's expectations, gaining voting consensus and \textit{truly} put fiscal reforms into effect seem to embed a trilemma in which only two-out-of-three could be achieved. In addition, trade wars and frictions between the US and China have not been helpful in coping with the virus, together with the "go-no-go" Brexit situation, the  internal strikes taking place after the George Floyd episode of end May and the confusion in the EU pre-COVID-19. 

Last but not least, corporate / household consumption and savings have always been central KPIs to be scrutinized during distress periods. During the first quarter of 2020, both American and European corporates have been taken precautionary decisions, such as limiting cash distribution (Figure \ref{fig:figure1} and \ref{fig:figure2}), delaying refinancing or M\&A activities and trying to cut costs not only in the operating finances, but also in the personnel space (e.g. despite being an extreme example, on June 2020 Lufhtansa estimated that 26,000 of their employees were risking their job). The forced lockdown and consequent working from home environment have corroded several businesses such as restaurants, automotive, building materials and more. These decisions could seem small from the outside, but interrupted steps like the aforementioned ones are generating a lethal macroeconomic chain mechanism: people saving money means more savings, and more savings means less money circulated in the economy; less money spent means reduced consumer credit demand, hence finally impacting the access to capital markets for the firms, forcing them to put activities on hold. Hence firms' losses. Hence layoffs. Hence people saving money. And the chain, wistfully, restarts. \\

\begin{figure}[H]
\centering
\begin{minipage}[b]{.5\textwidth}
\centering
\captionof{figure}{Percentage of Stoxx 600 companies cancelling dividends by sector.}
\includegraphics[scale=0.6]{DIV_STOXX.pdf}
\caption*{\textit{Source}: Bloomberg as of July 10, 2020.}
\label{fig:figure1}
\end{minipage}%
\hfill
\begin{minipage}[b]{.5\textwidth}
\centering
\captionof{figure}{Percentage of S\&P 500 companies cancelling dividends by sector.}
\includegraphics[scale=0.6]{DIV_SPX.pdf}
\caption*{\textit{Source}: Bloomberg as of July 10, 2020.}
\label{fig:figure2}
\end{minipage}
\end{figure}



Looking at the bright side, one of the most critical decision has been undertaken on May 27, when the European Commission announced the implementation of \euro{}750 billion stimulus aimed at mitigating the pandemic effect. The amount was increased to \euro{}1.35 trillion only one week later, with the ECB's Pandemic Emergency Purchase Programme (PEPP) put in place to guarantee the temporary repurchase of public and private assets / securities and inject money via a non-standard monetary policy \cite{pepp}.\footnote{Given the need of shifting from an office-based workplace to a smart-working mindset, investing in digitalization is another goal of the programme, so to promote more efficient IT services, internet connections and all of the tech-related realities that could boost the new approach to the job market.}

However, helps from the government are not enough to defuse hesitation and the aforementioned are only a few of the most compelling macro drivers that affect investors' decisions by strongly amplifying their skepticism.

In fact, economies suffering the aftermaths of the spread of the COVID-19 virus directly reflect on financial markets. We are currently experiencing unprecedented circumstances and whether markets will be able to recover from the consequences is yet to say.

Raising equity, for instance, has become progressively challenging. Focusing on the corporate side and in comparison with 2019, numbers and volumes of accelerated bookbuildings (ABBs) and convertible bonds (CBs) have overcome initial public offerings (IPOs) and rights issue (RIs)\footnote{\textit{Source}: Dealogic as of July 2020.}, due to two big reasons. Firstly, ABBs and CBs are faster, require less documentation and thus can be executed during the course of a day. Secondly and most importantly, due to how their process is organized, IPOs and RIs require the company to "stay-in-the-market" for a longer time (i.e. weeks for RIs and months for IPOs) and no firm is willing to be this exposed when the situation is the one we can appreciate since the beginning of 2020.

As a matter of fact, with the S\&P500 and the Stoxx600 touching their lows on March 23 (approximately -31\% and -33\% respectively\footnote{\textit{Source}: Bloomberg as of July 10, 2020.}), both the American and European economies are seriously threatened. In particular, as it is possible to appreciate from Figure \ref{fig:figure3}, all of the most important European stock markets have been hit by the COVID-19 crisis, with the fateful third week of March unanimously recognized as the most significant low.

%THE GRAPH MUST BE HERE
\begin{figure}[h]
\begin{center}
\caption{Most important European stock indices rebased to 100, performances from January 2020 to July 2020.}
\includegraphics[scale=0.6]{EUR_INDICES.pdf}
\caption*{\textit{Source}: Bloomberg as of July 10, 2020.}
\label{fig:figure3}
\end{center}
\end{figure}

\begin{figure}[h]
\centering
\begin{minipage}[b]{.45\textwidth}
\centering
\captionof{figure}{Performance of the American, Japanese, Hong Kong, Chinese and Emerging Markets indices rebased to 100, from January 2020 to July 2020.}
\includegraphics[scale=0.6]{US_EM_INDICES.pdf}
\caption*{\textit{Source}: Bloomberg as of July 10, 2020.}
\label{fig:figure4}
\end{minipage}%
\hfill
\begin{minipage}[b]{.45\textwidth}
\centering
\captionof{figure}{S\&P500 volatility analysis (10d, 30d, 60d, 90d) and CBOE's VIX Index's performance from January 2020 to July 2020.}
\includegraphics[scale=0.6]{SPX_VOL.pdf}
\caption*{\textit{Source}: Bloomberg as of July 10, 2020.}
\label{fig:figure5}
\end{minipage}
\end{figure}

The environment in the US and in the other major countries of the globe is not particularly better. Figure \ref{fig:figure4} below shows how the pattern is mirroring the European instance with the exception of the Shanghai index, which in turn is behaving 3-months-forwardly due to the different timeframe of the virus spread in China. As expected, emerging markets suffered the most during the last months: unorganized healthcare systems and the general state of economies such as Brazil or Mexico have made the recovery slower than their developed peers.

Concentrating on the United States, as previously mentioned the upcoming elections will have a major impact by confirming or reverting the positive signs displayed after mid June. Volatility has hit worrying levels (Figure \ref{fig:figure5}) and the Biden vs. Trump battle will not help in the process of stabilizing it. As of July, the Cboe's Vix remains 41 per cent above its historic average at nearly 28 and investors keep betting on US fluctuations, with the total equity and index option volumes booming in 2020 \cite{us_vol}.

When analyzing sector by sector, the situation does not change. Numbers for the lows of this crisis were hardly seen in the past. Of course, cyclical business have been affected more than others, with travel and leisure down to c.-55\% in mid March at the top (i.e. at the bottom) of this sad ranking. Oil \& gas has suffered a lot as well, while healthcare leads as "the-best-among-the-worst" together with utilities.\footnote{Respectively in mid March and mid July: O\&G c.-53\% and -34\%, Healthcare c.18\% and 2\%, Utilities c.-17\% and -1\%.}

Not only do Figures \ref{fig:figure6} and \ref{fig:figure7} below give us a hint about the gravity of the impact of the virus, but they also shed lights on a possible comeback. Tech, for example, seems to be back to normal levels, even if there is consensus that numbers could be misleading and that the shadow of a bubble dwells behind these figures. Yet investors are still wondering if the recovery will be in the face of an extreme V-shape, i.e. that, despite the slump caused by the virus, economies will be able to go back to the pre-crisis levels with a rapid pace, or more of a U-shape, meaning that markets have to navigate the pain for a while, before effectively recover. \\

\begin{figure}[H]
\centering
\begin{minipage}[b]{.45\textwidth}
\centering
\caption{Stoxx600 Sector Indices, sorted by performance.}
\includegraphics[scale=0.75]{STOXX600_SECTOR.pdf}
\caption*{\textit{Source}: Bloomberg as of July 10, 2020.}
\label{fig:figure6}
\end{minipage}%
\hfill
\begin{minipage}[b]{.45\textwidth}
\centering
\caption{S\&P500 Sector Indices, sorted by performance.}
\includegraphics[scale=0.75]{SPX_SECTOR.pdf}
\caption*{\textit{Source}: Bloomberg as of July 10, 2020.}
\label{fig:figure7}
\end{minipage}
\end{figure}

\newpage
\subsection{The research question}
Given the aforementioned framework, it is spontaneous to wonder if and how financial distress periods like this one influence the performance of factor-based allocation as well. What is the style exposure of the worst performing companies during these crises? Is there a cross-crisis common pattern related to what drives the returns of the stocks that suffered the most? Is there finally a winner in terms of performances between asset-based allocation and factor-based allocation?

In past literature there were rare cases of works were factor-investing's performance was analyzed during the several crises that markets faced in the last thirty years and, even when this was done, COVID-19 remains an uncovered ground. Being the context introduced, this is finally the investigation run in this work, i.e. trying to give an answer to the following research question:

\begin{center}\textit{"Factor-based investing: how do risk factors behave during crises?"}\end{center}

Firstly, a brief but detailed summary of the literature facing these topics will be introduced, because trying to express critical judgements in this field without having in mind the crucial results of the past would be Pareto-inefficient.\footnote{For those in love with Economics quotes, it would be indeed inefficient for a) the reader, because he/she would end up reviewing something that has no backbone at all, hence allocating his/her time to something neither valuable nor pleasant and b) for the writer, because this would not lead to productive effort whatsoever.} Secondly, some time will be spent discussing the different generated datasets. The data retrieval process plays a crucial role in the entire work, as the choice of different sources for the factors is personal and can change depending on the author (e.g.  factors could be selected from the very famous Kenneth French's database, but also using the EFT provided by MSCI that track stocks with specific characteristics, etc.).

Thirdly, the empirical and quantitative research will be conducted, i.e. the core of the entire work. In particular, this section is split into two complementary building blocks. The first one aims at describing the risk-return performance of the factors included in the analysis, mainly equity-oriented but also the fixed-income ones. The second focuses on equity styles, implementing an empirical exercise that could be summarized by the name of \textit{crisis regime analysis}. The aim of this part is trying to focus on how the returns of the worst performing stocks' return were explained by the different factors in different crises. This is something assimilable to the so-called event methodology, but one with key difference that is further described in the proper section. Lastly, results will be discussed in light of existing theory, with a concluding paragraph stating the main findings, but also exploring possible future studies.

% Literature Review paragraph:
\newpage
\section{Literature Review} \label{section:2}

While the first part of this section focuses on the pertinent results of the field, the second section introduces the most common factors, the rationale behind them and their empirical evidence. Indeed, these will be the factors used throughout the whole work.

\subsection{A glance at the past}

Despite being one of the most complicated aspects of modern finance, the principle beneath portfolio allocation is pretty simple: given the assumption that a specific and finite amount of capital is available to fund managers, they have to decide how much of this capital has to be invested in the possible instruments accessible in financial markets, so that an already predefined goal can be achieved. The first one to introduce a way to portfolio selection was Harry Markowitz, in 1952. When he introduced his acclaimed mean-variance (MV) optimization, Markowitz expressed his idea of how asset allocation is a two stages process. The first one starts with the studying of past performance of the securities and concludes with having an opinion on their future excess returns. The second one uses the output of the first stage as an input to generate portfolio allocation choices \cite{selection1952harry}.

However, the continuing development of financial markets resulted in a huge number of securities, thus leading to the adjustment of grouping securities together straight from the first stage (i.e. creating asset classes). In fact, running the MV optimization implicitly means estimating expected returns, variances and, called $N$ the number of available securities, $N (N-1) / 2$ correlations. Moreover, \citeA{ilmanen2003stock} studied the bond-stock correlation since 1926 and it was clear from that example that correlations do not show stability in behavior over time.\footnote{In particular, during 1929-1932, 1956-1965 and 1998-2015 average correlation between bonds and stocks was negative. Among the different causes, \citeA{ilmanen2003stock} stated that inflation played a big role, as bond-stock correlation tends to be limited during deflations and viceversa.} The dynamic nature of asset-classes correlation (together with the manifold degrees of freedom embedded in the estimation problem) led to the advancement of multifactor risk models.

With the Capital Asset Pricing Model (CAPM) of the 1960s, \citeA{sharpe1964capital}, \citeA{lintner1975valuation}, \citeA{treynor1961market} and \citeA{mossin1966equilibrium} attempted to explain how assets are priced in relation to their risk level. By owning risky securities, it was possible to generate excess returns because the market could act as the risk factor reflecting the intuition that investors have to bear a portion of risk if they want to see results. Thus, portfolio managers could achieve their goals by mimicking the portfolio of broad equity market cap indices (e.g. S\&P500 or Russell3000) and basically "owning" the market.

Less than twenty years later, \citeA{ross1976}, father of the Arbitrage Pricing Theory (APT), extended the idea of factors explaining returns, stating that these could be modeled as the contribution of one or more underlying factors:
\begin{equation} \label{equation1}
E(r_i - r_f) = \alpha+\beta_1 Factor_1+\beta_2 Factor_2+...+\beta_n Factor_n+\epsilon_i
\end{equation}

where $r_i$ is the return of security $i$, $r_f$ is the risk-free rate, $\alpha$ is the intercept of the regression, $\beta_j$ is the sensitivity (i.e. exposure) of the security $i$ to the $Factor_j$ and $\epsilon_i$ is the error term.

Despite contributing in a significant way, the APT did not provide any indication about which were the factors that could better describe excess returns. This major point was left to the empirical evidence supported by \citeA{fama1992cross}, who later introduced a Three Factors Model that, in addition to the market risk factor, used size and book-to-market equity (value) to describe cross-section of average stock returns.


\subsection{The risk factors}
Even if further studies juxtaposed more factors and theories to size and valuation, only some of them are typical of today's investment strategies. In particular, if one wanted to summarize the most utilized risk factors, it would be difficult not to cite the following:
\begin{itemize}

\item \textit{Size}: captures the excess returns obtained by purchasing stocks of small-cap companies, given that they generally outperformed large-cap stocks \cite{fama1992cross}.

\item \textit{Valuation}: captures the excess returns associated to stocks whose fundamental value appears cheap to the market. Such stocks are typically identified by looking at measurements such as P/E ratios and book-to-market value \cite{fama1992cross}.

\item \textit{Momentum}: captures the excess returns related to stocks exhibiting strong recent price performance. Companies performing well in the past are likely to do so in the future as well. While \citeA{jegadeesh1993returns} documented the impact of momentum on excess returns for the first time, \citeA{carhart1997persistence} added this factor in the Three Factors Model introduced by Fama and French. Furthermore, other studies investigated the relationship between momentum and valuation (\citeA{asness1997interaction}, \citeA{asness2013value}), trying to understand what are the main drivers behind their negative correlation.

\item \textit{Low Volatility}: stocks with lower standard deviation seem to generate significant risk-return performances. Firstly discussed by \citeA{black1972capital} and substantially controversial and anti-CAPM\footnote{One of the leading principles of CAPM's theory is that the higher the risk, the higher the return.}, this factor has always been the subject of behavioral studies, most of them leading to the conclusion that benchmarking has made low beta stocks look risky, hence more profitable.\footnote{The decrease in demand will generate a decrease in price and make the stocks more attractive. Others suggest that companies that are on the first page of major newspapers, i.e. typically very volatile stocks, tend to be overbought, leading to lower returns \cite{low_vol}.}

\item \textit{Profitability} and \textit{Investments}: \citeA{fama2015five} later introduced a Five Factors Model, comprehensive of the first three factors of their previous methodology (i.e. market, size, valuation) but extended to profitability (robust minus weak profitability firms) and investments (firms with conservative investments minus more aggressive ones).  These \textit{quality} indicators generally reflect the idea that companies whose analysts' forecasts are shinier, will have higher excess returns. \citeA{asness2019quality} advanced a model which is comprehensive of both investments and profitability, originally from 2014 and updated five years later (quality-minus-junk factor).

\end{itemize}

As discussed by \citeA{ESG_factors_Blackrock} in one of the latest articles of his factor commentary, all of these factors satisfy four criteria:
\begin{enumerate}
\item They all have a solid academic background justifying the equity premium associated to each of them;
\item Years and years of empirical evidence support these justifications;
\item They can be implemented at scale;
\item They exhibit low correlations with other factors, adding value to market cap indices.
\end{enumerate}

Point number 4. represents the highest obstacle to be overcome when it comes to adding new factors in the current literature, because managers and investors hope for diversification.\footnote{The idea of the Occam's Razor (i.e. it is not efficient to do something with multiple entities, if you could achieve the same results using fewer of them) perfectly fits here, because adding a new driver which is positively correlated with the already existing ones, would not help the fund manager to better explain excess returns.}

Several studies are now giving credit to the possibility of fixed-income style investing, thus being the next frontier in the field. Notice that the above mentioned factors hold also for this world, because what actually changes is the proxy used to capture the factors themselves. Following a recent study conducted by a division of S\&P Global \cite{fixed-income-investing} and coherently with two of the leading reference papers of this work (i.e. \citeA{bender2010portfolio} and \citeA{idzorek2013factor}), the following three fixed-income proxies will be discussed, to analyze risk-factors in this environment as well:

\begin{itemize}
\item \textit{Credit Spread}: The credit spread factor is identified as a portfolio long in investment grade corporate bonds and short in Treasury issued debt. The rationale behind this factor is that "by investing in corporate bonds, investors collect this risk premium over time for bearing the credit risk associated with these securities" \cite{fixed-income-investing-fidelity}, i.e. the risk that companies will not respect their principal / interest obligations. This spread is a common measure of valuation of investment-grade corporate bonds.

\item \textit{High Yield Spread}: The high yield spread is defined as a portfolio long in high yield corporate bonds and short in the generic corporate bond world (as briefly introduced before and as deeply explained in \textit{Section \ref{factors_definitions_dataset} - \nameref{factors_definitions_dataset}}, this is the way all factors are "isolated", hence identified). This proxy is a measure for the quality factor, as usually lower quality bonds intrinsically carry higher yields than better quality ones (to compensate the distance in quality itself).

\item \textit{Term Spread}: The term spread (sometimes called duration times spread, DTS) is identified as a portfolio long in protracted maturity government debt (say 20 or 10 years) and short in brief maturity government debt (say 1-3 years). In fact, "empirical and academic research studies have shown that over a long investment horizon, longer-term bonds, on average, earn higher returns than shorter-term bonds" \cite{fixed-income-investing}. Volatility represents the factor captured by this proxy, as duration itself is nothing but a measure of the sensitivity of an instrument price to changes in interest rates.
\end{itemize}

One last point has to be underlined before entering in the core part of the analysis: if factor investing has demonstrated superior returns both from a theoretical and empirical standpoint, why aren't the classic investment strategies drastically replaced by equity and fixed income styles? The answer relies on the fact that "in order to achieve proper risk balance and attain the high returns and low correlation properties investors seek, style investing requires the \textit{three dirty words in finance}: leverage, short-selling and derivatives. For investors willing (and able) to use these risky tools, there is the potential for huge rewards in terms of better and more stable returns" \cite{asness2015investing}.

%%%%%%%%NOT CITED YET%%%%%%
% \cite{bender2013foundations}


% Methodology paragraph:
\newpage
\section{Methodology} \label{section:4}
First of all, the datasets and their construction are introduced, in addition to how the factor themselves are being defined. Following is a subsection on how the long term selected period was decomposed into six major crises, mentioning the technical details about how the analysis is going to be run from a temporal standpoint.

The subsequent part is then split into two complementary blocks. The first one follows \citeA{bender2010portfolio}, as it is a reconstruction of the risk-return performances of both asset classes and factors exposures over the different turbulence periods. In order to avoid long-winded excesses, only the most important and fascinating findings will be discussed in this first subsection. The second subsection focuses on an empirical exercise which intends to analyze what were the main factor exposures that the ex-post worst performing companies were bearing before entering the crises.

\subsection{Datasets and factors definition} \label{factors_definitions_dataset}
In order to perform a more comprehensive analysis, two different datasets were equally \textit{constructed}\footnote{As often in empirical papers, building the datasets in a coherent and robust manner was one of the most compelling part of the entire work.} first and utilized then, datasets that will be needed throughout the whole methodology section. As previously mentioned, the choice of which source to use is fundamental when dealing with factors. By examining past literature, in fact, it results difficult to find common consensus on which are the best indices to include in this kind of analysis. Of course, the expected results is to find no surprising distance between the two datasets given that, regardless of the source, they both intend to describe the same underlying factors.

Firstly, the emphasis has to be put on the equity-oriented premia. The first dataset follows the work from \citeA{bender2010portfolio} excluding the strategy premia, because including the Merger Arbitrage premium or Currency Carry Trade would result in a too dispersive study (especially given the choice of building two datasets). Nonetheless, the analysis run in the methodology section could be adapted to the inclusion of those factors as well, extending their work to other ten years of data. This dataset will be denoted as \textit{Dataset A} throughout the work.

The second basket of data will be based on the work from \citeA{idzorek2013factor}. From now on, this dataset will be denoted as \textit{Dataset B}. It is important to notice that the different sources used for Dataset B implicitly shifts the focus from a world-based standpoint to a US-emphasized analysis, as most of the considered variables are US focused. This is coherent with the expectation of limited differences in the usage of the two datasets, as usually US indices are considered to be good proxies for world based ones.

However, both these two pieces of literature privilege some factors at the expense of others, hence creating discrepancies in the construction of both Dataset A and B. For example, \citeA{idzorek2013factor} does not consider Momentum at all. Even more evidently, both works do not consider Low Volatility and Quality in their exercises. Hence, whenever these instances happen, sources from different papers will be merged, so to have as far distance as possible between the two datasets.

Secondly, walking the same path followed in the \textit{\nameref{section:2} Section}, fixed income-oriented factors are specified. These will mainly be US based for both datasets, because they represent a fairly good proxy for the Credit Spread, High Yield Spread and Term Spread in both Datasets.

The broad datasets context being introduced, it must be remembered that Equation \ref{equation1} describes the process of explaining returns as the involvement of different factors justifying the price movements (thus, returns), with different sensitivities to those factors. However, how are these risk premia captured from a practical standpoint? The answer relies on the \textit{isolation of the exposure to that particular risk premium thanks to a long-short portfolio aimed at encapsulating the nature of the premium only}, without further sources of return in it. Hence (and following what implemented in the reference papers for this work\footnote{However, notice that only Market, Size, Value and Momentum are defined in this way in the papers, as Low Volatility and Quality are not considered. The implicit assumption of this work is that this way of construction holds also for these two factors. Ideally, one would be to build the portfolios on a company-by-company basis but, in order to avoid incurring in possible biases when doing so, the more direct long-short indices approach was preferred. This will be one big point of further studying, therefore inserted in \textit{Section \ref{limitations} - \nameref{limitations}.}}), factors are obtained by going long one dimension while shorting the opposite one. By doing this, we implicitly implement a self-financing strategy where the factor-tilted long position will be financially sustained by the shorted neutral exposure. For example, by going long the MSCI World Minimum Volatility index we aim to reflect the performance characteristics of a minimum variance strategy applied to the MSCI large and mid cap equity universe across 23 Developed Markets countries\footnote{\textit{Source}: Bloomberg description of the MSCI World Minimum Volatility Index.}; on the other hand, by simultaneously shorting the MSCI World Index we are eliminating market exposure and capturing the Low Volatility premium only. The same reasoning holds for both equity-based and fixed-income-oriented factors. Note that the best scenario would be buying the factor-exposed index by shorting the opposite exposure index. This was implemented whenever indices were available in Bloomberg. If not, the short position would simply be the broad index from which the long position is built.

The sum of the contributions of asset class and style risk exposures will define the part of returns explained by risk premia. Regarding the former, the final decision was to include a common basket of data for the asset class returns, as the compensation investors earn in excess of risk free rate does not have significant swings in meaning when considering two different sources. Any other possible source of return, is generally attributable to the manager's capacity to generate superior performance, i.e. the fund manager's alpha.

In order to facilitate the understanding of the following section, Table \ref{factors_datasets_table} intends to summarize all the used risk premia, indicating which datasets they constitute, which are the long-short dimensions employed to build them and the piece of literature they come from.

Finally, notice that all of the aforementioned indices are "net total return", i.e. the underlying assumption is that dividends are reinvested in the companies composing the indices, and the effect is net of taxes. These types of proxies have been preferred with respect to (a) the mere "price indices", because these do not consider dividends at all (i.e. the reinvestment hypothesis would not hold) and to (b) the "gross total return indices", as these do not consider the tax effect (which cannot be believed as marginal). The only exception relates to asset class exposures, for which price indices have been used.\footnote{Unfortunately, the MSCI Emerging Markets Net Total Return is only available from 01/01/1999, i.e. two years after the selected starting date. However, being only single entities (i.e. not a result of long-short portfolio), price indices offer the same information in terms of returns, with verified negligible differences vs. the net total return indices.} Of course, the analysis, methodology and results would hold the same if (a) or (b) were to be used, with results slightly changing to reflect these variations. \\

\begin{table}[H]
\caption{Summary of the indices used to determine the risk factors studied in both Dataset A and B.} \label{factors_datasets_table}
\resizebox{\textwidth}{!}{%
\begin{tabular}{@{}lllll@{}}
\toprule
                                    & \multicolumn{2}{c}{\textit{\textbf{Dataset A}}}                          & \multicolumn{2}{c}{\textbf{Dataset B}}                                  \\ \midrule
                                    & \multicolumn{1}{c}{Long Position}  & \multicolumn{1}{c}{Short Position}  & \multicolumn{1}{c}{Long Position}  & \multicolumn{1}{c}{Short Position} \\ \midrule
\textit{1. Equity-oriented premia}         &                                    &                                     &                                    &                                    \\ \midrule
Market                              & MSCI World                         & Barclays US T. Bill Index           & Russell 3000 Index\textsuperscript{(2)}                 & Barclays US T. Bill Index          \\
Value                               & MSCI World Value\textsuperscript{(1)}                   & MSCI World Growth\textsuperscript{(1)}                   & Russell 3000 Value Index\textsuperscript{(2)}           & Russell 3000 Growth Index\textsuperscript{(2)}          \\
Size                                & MSCI AC World Small Cap\textsuperscript{(1)}            & MSCI AC World Large Cap\textsuperscript{(1)}             & Russell 2000 Index\textsuperscript{(2)}                 & Russell 1000 Index\textsuperscript{(2)}                 \\
Momentum                            & MSCI World Momentum\textsuperscript{(1)}                & MSCI World\textsuperscript{(1)}                          & MSCI USA Momentum                  & MSCI USA                           \\
Low Volatility                      & MSCI World Minimum Volatility      & MSCI World                          & MSCI USA Minimum Volatility        & MSCI USA                           \\
Quality                             & MSCI World Quality                 & MSCI World                          & MSCI USA Quality                   & MSCI USA                           \\ \midrule
\textit{2.   F.I.-oriented premia} & \textit{}                          &                                     &                                    &                                    \\ \midrule
Credit Spread                       & Citigroup USBIG Corp (AAA/AA)\textsuperscript{(1)}      & Citigroup USBIG Treas./Gov.\textsuperscript{(1)}         & Barclays U.S. Credit\textsuperscript{(2)}               & Barclays U.S. Treasury\textsuperscript{(2)}             \\
High Yield Spread                   & Citigroup High Yield Market\textsuperscript{(1)}        & Citigroup USBIG Corp (AAA/AA)\textsuperscript{(1)}       & Barclays High Yield Market         & Barclays U.S. Credit               \\
Term Spread                         & Citigroup USBIG Treas. 10+ Y\textsuperscript{(1)}       & Citigroup USBIG Treas. 1-3 Y\textsuperscript{(1)}        & Barclays U.S. Treas. 20+ Y\textsuperscript{(2)}         & Barclays U.S. Treas. 1-3 Y\textsuperscript{(2)}         \\ \midrule
\textit{3.   Asset Class (Equity)\textsuperscript{(3)}}  & Description                        &                                     &                                    &                                    \\ \midrule
MSCI EAFE\textsuperscript{(1)}                           & \multicolumn{4}{l}{Captures large and mid cap representation across 21 Developed Markets countries around the world, excluding the US and Canada.} \\
MSCI Japan\textsuperscript{(1)}                          & \multicolumn{4}{l}{Designed to capture 85\% of the free float-adjusted market capitalization in Japan (large and mid cap).}                        \\
MSCI USA\textsuperscript{(1)}                            & \multicolumn{4}{l}{Designed to capture 85\% of the free float-adjusted market capitalization in the US (large and mid cap).}                       \\
MSCI EM\textsuperscript{(1)}                             & \multicolumn{4}{l}{Captures large and mid cap representation across 26 Emerging Markets (EM) countries.}                                           \\ \midrule
4.   \textit{Asset Class (F.I.)\textsuperscript{(3)}}             & Description                        &                                     &                                    &                                    \\ \midrule
Citi World Bond Index               & \multicolumn{4}{l}{A broad index providing exposure to the global sovereign fixed income market.}                                                  \\ \bottomrule
\end{tabular}}
\bigskip
\caption*{\textit{Notes}: (1) \protect\citeA{bender2010portfolio}; (2) \protect\citeA{idzorek2013factor}; other factors have been identified either through author's decision or a combination of the latter and the previously cited papers (e.g.: \protect\citeA{idzorek2013factor} uses the Russell 3000 Index to identify the market, but subtracts the Citigroup 3-months T. Bill, which is in turn not considered here); (3) common to both datasets.}
\end{table}

\subsection{Financial distress periods: timeframe subdivision} \label{section3.2}

The availabilities of the aforementioned indices share a common initial root date aging back to January 1, 1997. This is  considered as the starting point throughout the whole work. On the other hand, the analysis is concluded on the July 10, 2020, date on which both Dataset A and B were finalized. Despite sharing this common starting date (in terms of data retrivability), the selected indices displayed missing data in the time series analysis. Whenever this was the case, the missing points were linearly (hence, temporally) interpolated, practice that both past literature and Bloomberg itself often embrace.

This section is crucial to be highlighted because, in order to better understand the performance of the factors, a long term analysis has to be conducted first, in addition to the simple subdivision of the crises. The latter are then briefly summarized in Table \ref{crises_table} below, in part following \citeA{bender2010portfolio} (slightly adapting the different selected time horizons) and comprehensive of the COVID-19 period as well, differentiating contribution of this work. Notice that it could happen that an overlapping period is considered when analyzing two different crises. This is the case for three reasons. First of all, the interconnection among financial markets has made them stronger, but more exposed to a single breaking point, generating a waterfall effect lasting years. More practically for the purpose of the work:
\begin{enumerate}
\item It must be underlined that isolating the financial distress period itself is indeed crucial, but so is the availability of at least two years of daily returns, to offer a minimum degree of robustness in the results.
\item It is easy to determine the exact length of a crisis only a posteriori, but never a priori, i.e. when the crisis is being lived.
\end{enumerate}

Hence, what will be actually analyzed is the annualized risk-return performance of the considered financial distress period, though leaving the door opened to the successive recovery (of course, when this was not experienced too far away in the imminent future, say not lagged more than two years after the ending date). This makes the discussed considerations replicable in future crises, properly because we don't have perfect foresight when experiencing the turmoil itself. The only exception to this rule was made for the COVID-19 pandemic, given that considering returns from July 2018 to July 2020 would mean completely losing the isolation aspect of the crisis. \\

\begin{table}[H]
\centering
\caption{Summary of the financial distress periods considered throughout the thesis.}
\label{crises_table}
\resizebox{.5\textwidth}{!}{%
\begin{tabular}{@{}lcc@{}}
\toprule
\textbf{Crisis}             & \textbf{Start Date} & \textbf{End Date} \\ \midrule
Asian Crisis 1997           & 1-Jan-97            & 31-Dec-98          \\
August 1998 turmoil         & 1-Aug-98            & 31-Jul-00          \\
Dot-com Bubble \& 11 Sept 2001 & 1-Jan-00            & 31-Dec-02         \\
2008 Financial Crisis       & 31-Oct-07           & 31-Oct-09         \\
(Peak of) European Debt Crisis        & 1-Jan-10            & 31-Dec-12         \\
COVID-19 Pandemic           & 1-Jan-20            & 10-Jul-20         \\ \bottomrule
\end{tabular}}
\bigskip
\caption*{\textit{Notes}: No financial distress period has been considered between 2013 and 2020 because, despite the Trade War and Brexit caused significance turbulences, respecting the two years data self-imposed constraint during this long period would inevitably considering flourishing years as well, risking to bias the analysis.}
\end{table}

\subsection{Risk-return performance}
Analyzing how both asset class exposures and style premia have performed during the different financial crises will help to understand more about their behavior in a time series context. Despite annualized return and volatility remain of big interest, potential diversification effects are being tackled, as it is well known that correlations across risk factors are usually lower than across asset classes (both \citeA{page2011invited} and \citeA{bender2010portfolio}). 

Regarding asset classes, the process to calculate the needed metrics was straightforward, being those exposed to a single index only. To compute the relevant measures in the style world, instead, a three steps approach was implemented (intended to avoid Spurious outcomes). First, daily returns for the individual long and short positions were computed. Then, actual long-short portfolios were generated by subtracting the two components. Finally, the geometric average of returns was calculated (daily average first and annualized average later)\footnote{Notice that the GEOMEAN() formula in Excel does not work with more than 3,500 data points. The solution resides in using a log transformation, as this is equivalent to taking the exponent of the average of the values' logs (proof in Appendix).}, as long as the annualized volatility (using the 252 market days convention, to be coherent in these terms) and the annualized sharpe ratio. In addition, parametric and historical 95\% and 99\% Value at Risk (VaR) have been computed, to provide another measure of risk on the side of the simple standard deviation. Given that VaR answers to the question of what the probability of losing more than \textit{p} (with a certain confidence level) is, but does not consider the magnitude of the losses when they occur, the Expected Shortfall (ES, i.e. average of the losses exceeding the VaR) has been computed in both VaR types and confidence levels.\footnote{Of course, these represent point in time values which have to be considered carefully. A more extended analysis would be to study both VaR and ES on a rolling basis (i.e. 100 or 500 days), maybe considering Exponentially Weighting Moving Average volatility as well, but the scope of this work does not vastly extend to risk management topics intentionally.} Due to space reasons, 95\% confidence level results only are reported. For the same reasons, the correlation table by crisis for Dataset B has been left out on purpose, as results were coherent with the more general case. Moreover,  both Value at Risk and ES will be expressed in absolute value whenever not specified otherwise.

The entire analysis is first run during the long term spectrum and then faced on a crisis-by-crisis basis, with a particular focus on the COVID-19 Pandemic. Again, the goal of this section is not to linger on every single outcome, rather to underline the most interesting results in the bigger picture. 

\subsection{Crisis regime analysis methodology}

The second core part regards an empirical exercise which could be defined as a \textit{crisis regime analysis}. This  can be assimilated to the so-called \textit{event study methodology} but with a significant conceptual difference that, in order to be clarified, requires the understanding of what is usually meant by event study. One of the first works to discuss this approach was the study from \citeA{mackinlay1997event}, in his paper providing an example of how quarterly earning announcements influence firms market values.

In general terms, an event study allows us to measure the impact of specific events on topics of interest. Fields of application are several, ranging from economics to law studies, usually with the common goal of understanding how specific circumstances can shape prices and equity. In finance and similar fields, "the usefulness of such a study comes from the fact that, given rationality in the marketplace, the effects of an event will be reflected immediately in security prices. Thus a measure of the event's economic impact can be constructed using security prices observed over a relatively short time period" \cite{mackinlay1997event}. This quote contains the key difference between a classical event study and the approach undertaken in this work, i.e. \textit{time dilation}: while the former focuses on a short term and immediate effect, the methodology of this project studies the event on an extended temporal basis.

In fact, in the instance of this work, the event study is split into the following steps:
\begin{enumerate}
\item Decompose the different crises as done in \textit{Subsection \ref{section3.2} - \nameref{section3.2}}.
\item Select a sample of the worst performing companies during those distress periods.
\item Analyze the excess returns of those stocks during a prior window with respect to the timespans selected in step 1.
\end{enumerate}

In order to select the worst performing stocks in step 2, the iShare MSCI World ETF will be used. Even though using the ETF is not 100\% accurate, this still remains the best proxy, as the iShare MSCI World aims at tracking the MSCI World performance and hence its constituents.\footnote{Bloomberg requires special authorizations to have the time series member ranked returns for the MSCI World itself. This is also why the ETF was preferred.} Regarding step 3, instead, the followed procedure was to regress the excess returns of the worst performing stocks on the equity styles\footnote{Once again, only results involving Dataset A will be reported for reasons of space, as the regressions showed similar patterns when focusing on the US thanks to Dataset B.}, in order to understand how the stocks that have granted the worst returns during a specific crisis were actually exposed to risk factors, \textit{before} entering the turbulence. This process also allows to study possible cross-crisis patterns and indirectly predict which \textit{could} be the most likely exposure characterizing the worst performing firms, if similar circumstances were to occur again.

Given that this procedure engages with lagged windows, it is possible to use even more than the 2 years daily data. In particular, 3 years of market daily data were employed to build the lagged windows. However, as aforementioned, the common starting date for all factors' availability is January 1, 1997, making the event study method not applicable for the Asian Crisis\footnote{How can the previous 3 years of excess returns of the worst performing companies be regressed, if data for factors themselves is available from January 1997?} and imposing an implicit 1.5 years constraint for the 1998 turmoil, as the selected starting date for this crisis is August '98. Regarding the Asian tumults, the static analysis will be conducted in any case, i.e. the excess returns of the worst performing stocks will be regressed on the factors' returns over the same period. Notice that this will not only mean ad-hoc results have to be faced (as the other crises are based on the shifted windows, while this one is not), but also that prediction and cross-crisis judgements must be carefully considered when involving this particular distress period.

Moreover, before selecting the worst 50 stocks during the crises, a data availability test was run for those stocks, test that aimed at excluding firms if few daily returns were available. Of course, the exclusion ratio was kept as low as possible to avoid selection bias, almost tending to 0\% as far as more recent crises were concerned.\footnote{For the most recent COVID-19, European Debt Crisis, 2008 and Dot-com Bubble data was easily retrievable with almost zero companies excluded. This process was slightly more complex for the Asian Crisis and the 98 turmoil, as constituents at the bottom of the iShare MSCI World ETF were missing a non-negligible amount of daily datapoints.}

Therefore, the following several regressions will be run, in parallel with Equation \ref{equation1}:
\begin{equation} \label{equation2}
\begin{split}
E(r_i - r_f) &= \alpha+\beta_1 Market +\beta_2 Valuation + \beta_3 Size + \\
&  + \beta_4 Momentum + \beta_5 LowVolatility +\beta_6 Quality + \epsilon_i \\
& \forall  i=1,...,50 \text{ stocks and } \forall \text{ considered crisis in Table \ref{crises_table}.}
\end{split}
\end{equation}

A small script written in Stata will be used in order to perform the above in an automatic way and export the most relevant metrics in Excel (i.e. coefficients' estimates, their standard error, R-squared and degrees of freedom).\footnote{Another small piece of code was implemented in order to store t-statistics and p-values as well, given that Stata does not save them as variables. The complete script is present in Appendix.}

% Results paragraph:
\newpage
\section{Results \& Discussion}

\subsection{Long term spectrum - 1\textsuperscript{st} January 1997 to 10\textsuperscript{th} July 2020} \label{whole_period}
This section outlines the risk-return conduct of the risk premia throughout the whole selected timeframe, i.e. almost twenty-four years of daily data. Decomposing these trends by studying the financial crises individually will help to lighten most of the arisen doubts. \\

% If you use beamer only pass "xcolor=table" option, i.e. \documentclass[xcolor=table]{beamer}
\begin{table}[H]
\caption{Risk-return performance analysis of Datasets A and B's exposures, January 1997 - July 2020.} \label{1997_2020_RR_table}
\resizebox{\textwidth}{!}{%
\begin{tabular}{@{}llccccccc@{}}
\toprule
Exposure                      & \begin{tabular}[c]{@{}l@{}}Risk\\ Premium\end{tabular} & \begin{tabular}[c]{@{}c@{}}Annualized\\ Return\end{tabular} & \begin{tabular}[c]{@{}c@{}}Annualized\\ Volatility\end{tabular} & \begin{tabular}[c]{@{}c@{}}Sharpe\\ Ratio\end{tabular} & \begin{tabular}[c]{@{}c@{}}Parametric\\ 95\% VaR\end{tabular} & \begin{tabular}[c]{@{}c@{}}Parametric\\ 95\% ES\end{tabular} & \begin{tabular}[c]{@{}c@{}}Historical\\ 95\% VaR\end{tabular} & \begin{tabular}[c]{@{}c@{}}Historical\\ 95\%ES\end{tabular} \\ \midrule
\textit{Asset Class (Equity)} & MSCI EAFE                                              & 1.76\%                                                      & 17.26\%                                                         & 0.10                                                   & 1.78\%                                                        & 2.69\%                                                       & 1.70\%                                                        & 2.61\%                                                      \\
                              & MSCI JAPAN                                             & 0.13\%                                                      & 21.20\%                                                         & 0.01                                                   & 2.19\%                                                        & 3.23\%                                                       & 2.11\%                                                        & 3.11\%                                                      \\
                              & MSCI USA                                               & 6.23\%                                                      & 19.41\%                                                         & 0.32                                                   & 1.98\%                                                        & 3.10\%                                                       & 1.85\%                                                        & 2.95\%                                                      \\
                              & MSCI EM                                                & 3.38\%                                                      & 18.94\%                                                         & 0.18                                                   & 1.94\%                                                        & 3.03\%                                                       & 1.87\%                                                        & 2.91\%                                                      \\ \midrule
\textit{Asset Class (F.I.)}   & Citi World Bond                                        & 4.09\%                                                      & 6.47\%                                                          & 0.63                                                   & 0.65\%                                                        & 0.92\%                                                       & 0.63\%                                                        & 0.89\%                                                      \\ \midrule
\textit{Style (Equity)}       & Market                                                 & 3.83\%                                                      & 15.61\%                                                         & 0.25                                                   & 1.60\%                                                        & 2.54\%                                                       & 1.48\%                                                        & 2.42\%                                                      \\
                              &                                                        & {\color[HTML]{303498} 5.89\%}                               & {\color[HTML]{303498} 19.48\%}                                  & {\color[HTML]{303498} 0.30}                            & {\color[HTML]{303498} 1.99\%}                                 & {\color[HTML]{303498} 3.06\%}                                & {\color[HTML]{303498} 1.86\%}                                 & {\color[HTML]{303498} 2.95\%}                               \\
                              & Valuation                                              & (1.92\%)                                                     & 6.43\%                                                          & (0.30)                                                 & 0.67\%                                                        & 1.04\%                                                       & 0.58\%                                                        & 0.94\%                                                      \\
                              &                                                        & {\color[HTML]{303498} (1.92\%)}                             & {\color[HTML]{303498} 9.76\%}                                   & {\color[HTML]{303498} (0.20)}                          & {\color[HTML]{303498} 1.02\%}                                 & {\color[HTML]{303498} 1.73\%}                                & {\color[HTML]{303498} 0.84\%}                                 & {\color[HTML]{303498} 1.46\%}                               \\
                              & Size                                                   & 0.81\%                                                      & 6.24\%                                                          & 0.13                                                   & 0.64\%                                                        & 0.96\%                                                       & 0.63\%                                                        & 0.96\%                                                      \\
                              &                                                        & {\color[HTML]{303498} (0.74\%)}                             & {\color[HTML]{303498} 10.33\%}                                  & {\color[HTML]{303498} (0.07)}                          & {\color[HTML]{303498} 1.07\%}                                 & {\color[HTML]{303498} 1.52\%}                                & {\color[HTML]{303498} 1.00\%}                                 & {\color[HTML]{303498} 1.45\%}                               \\
                              & Momentum                                               & 3.35\%                                                      & 8.55\%                                                          & 0.39                                                   & 0.87\%                                                        & 1.39\%                                                       & 0.85\%                                                        & 1.35\%                                                      \\
                              &                                                        & {\color[HTML]{303498} 3.27\%}                               & {\color[HTML]{303498} 11.80\%}                                  & {\color[HTML]{303498} 0.28}                            & {\color[HTML]{303498} 1.21\%}                                 & {\color[HTML]{303498} 1.99\%}                                & {\color[HTML]{303498} 1.08\%}                                 & {\color[HTML]{303498} 1.86\%}                               \\
                              & Low Volatility                                         & 0.13\%                                                      & 7.23\%                                                          & 0.02                                                   & 0.75\%                                                        & 1.13\%                                                       & 0.70\%                                                        & 1.07\%                                                      \\
                              &                                                        & {\color[HTML]{303498} (0.20\%)}                             & {\color[HTML]{303498} 8.47\%}                                   & {\color[HTML]{303498} (0.02)}                          & {\color[HTML]{303498} 0.88\%}                                 & {\color[HTML]{303498} 1.41\%}                                & {\color[HTML]{303498} 0.79\%}                                 & {\color[HTML]{303498} 1.28\%}                               \\
                              & Quality                                                & 2.54\%                                                      & 6.61\%                                                          & 0.38                                                   & 0.67\%                                                        & 1.17\%                                                       & 0.53\%                                                        & 0.97\%                                                      \\
                              &                                                        & {\color[HTML]{303498} 1.47\%}                               & {\color[HTML]{303498} 9.31\%}                                   & {\color[HTML]{303498} 0.16}                            & {\color[HTML]{303498} 0.96\%}                                 & {\color[HTML]{303498} 1.68\%}                                & {\color[HTML]{303498} 0.75\%}                                 & {\color[HTML]{303498} 1.48\%}                               \\ \midrule
\textit{Style (F.I.)}         & Credit Spread                                          & 0.28\%                                                      & 1.42\%                                                          & 0.20                                                   & 0.15\%                                                        & 0.26\%                                                       & 0.11\%                                                        & 0.21\%                                                      \\
                              &                                                        & {\color[HTML]{303498} 1.01\%}                               & {\color[HTML]{303498} 2.55\%}                                   & {\color[HTML]{303498} 0.40}                            & {\color[HTML]{303498} 0.26\%}                                 & {\color[HTML]{303498} 0.52\%}                                & {\color[HTML]{303498} 0.18\%}                                 & {\color[HTML]{303498} 0.39\%}                               \\
                              & High Yield Spread                                      & 0.92\%                                                      & 6.03\%                                                          & 0.15                                                   & 0.62\%                                                        & 1.06\%                                                       & 0.53\%                                                        & 0.93\%                                                      \\
                              &                                                        & {\color[HTML]{303498} 0.27\%}                               & {\color[HTML]{303498} 6.00\%}                                   & {\color[HTML]{303498} 0.05}                            & {\color[HTML]{303498} 0.62\%}                                 & {\color[HTML]{303498} 0.97\%}                                & {\color[HTML]{303498} 0.56\%}                                 & {\color[HTML]{303498} 0.90\%}                               \\
                              & Term Spread                                            & 4.49\%                                                      & 10.45\%                                                         & 0.43                                                   & 1.06\%                                                        & 1.51\%                                                       & 1.04\%                                                        & 1.50\%                                                      \\
                              &                                                        & {\color[HTML]{303498} 4.61\%}                               & {\color[HTML]{303498} 11.75\%}                                  & {\color[HTML]{303498} 0.39}                            & {\color[HTML]{303498} 1.20\%}                                 & {\color[HTML]{303498} 1.72\%}                                & {\color[HTML]{303498} 1.16\%}                                 & {\color[HTML]{303498} 1.68\%}                               \\ \bottomrule
\end{tabular}}
\medskip
\caption*{\textit{Notes}: For style factors: Dataset A values in black, Dataset B values in blue.}
\end{table}

%%%

\begin{table}[H]
\caption{Correlations of Datasets A and B's style and asset class premia, January 1997 - July 2020.} \label{1997_2020_corr_table}
\resizebox{\textwidth}{!}{%
\begin{tabular}{@{}lccccccccc@{}}
\toprule
\multicolumn{1}{c}{}  & Market                                 & Valuation                              & Size                                   & Momentum                               & \begin{tabular}[c]{@{}c@{}}Low Volatility\end{tabular} & Quality                                & \begin{tabular}[c]{@{}c@{}}Credit Spread\end{tabular} & \begin{tabular}[c]{@{}c@{}}H.Y. Spread\end{tabular} & \begin{tabular}[c]{@{}c@{}}Term Spread\end{tabular} \\ \midrule
Market                & 1.00                                   & \textbf{}                              & \textbf{}                              & \textbf{}                              & \textbf{}                                                & \textbf{}                              & \textbf{}                                               & \textbf{}                                                   & \textbf{}                                             \\
                      & {\color[HTML]{303498} 1.00}            & \multicolumn{1}{l}{}                   & \multicolumn{1}{l}{}                   & \multicolumn{1}{l}{}                   & \multicolumn{1}{l}{}                                     & \multicolumn{1}{l}{}                   & \multicolumn{1}{l}{}                                    & \multicolumn{1}{l}{}                                        & \multicolumn{1}{l}{}                                  \\
Valuation             & \textbf{0.06}                          & 1.00                                   & \textbf{}                              & \textbf{}                              & \textbf{}                                                & \textbf{}                              & \textbf{}                                               & \textbf{}                                                   & \textbf{}                                             \\
                      & {\color[HTML]{303498} \textbf{(0.14)}} & {\color[HTML]{303498} 1.00}            & \multicolumn{1}{l}{}                   & \multicolumn{1}{l}{}                   & \multicolumn{1}{l}{}                                     & \multicolumn{1}{l}{}                   & \multicolumn{1}{l}{}                                    & \multicolumn{1}{l}{}                                        & \multicolumn{1}{l}{}                                  \\
Size                  & \textbf{(0.18)}                        & \textbf{0.06}                          & 1.00                                   & \textbf{}                              & \textbf{}                                                & \textbf{}                              & \textbf{}                                               & \textbf{}                                                   & \textbf{}                                             \\
                      & {\color[HTML]{303498} \textbf{0.19}}   & {\color[HTML]{303498} \textbf{0.03}}   & {\color[HTML]{303498} 1.00}            & \multicolumn{1}{l}{}                   & \multicolumn{1}{l}{}                                     & \multicolumn{1}{l}{}                   & \multicolumn{1}{l}{}                                    & \multicolumn{1}{l}{}                                        & \multicolumn{1}{l}{}                                  \\
Momentum              & \textbf{(0.13)}                        & \textbf{(0.37)}                        & \textbf{(0.04)}                        & 1.00                                   & \textbf{}                                                & \textbf{}                              & \textbf{}                                               & \textbf{}                                                   & \textbf{}                                             \\
                      & {\color[HTML]{303498} 0.22}            & {\color[HTML]{303498} \textbf{(0.39)}} & {\color[HTML]{303498} \textbf{(0.10)}} & {\color[HTML]{303498} 1.00}            & \multicolumn{1}{l}{}                                     & \multicolumn{1}{l}{}                   & \multicolumn{1}{l}{}                                    & \multicolumn{1}{l}{}                                        & \multicolumn{1}{l}{}                                  \\
Low Volatility        & \textbf{(0.70)}                        & \textbf{0.10}                          & \textbf{0.05}                          & 0.35                                   & 1.00                                                     & \textbf{}                              & \textbf{}                                               & \textbf{}                                                   & \textbf{}                                             \\
                      & {\color[HTML]{303498} \textbf{(0.14)}} & {\color[HTML]{303498} \textbf{(0.01)}} & {\color[HTML]{303498} \textbf{(0.28)}} & {\color[HTML]{303498} 0.64}            & {\color[HTML]{303498} 1.00}                              & \multicolumn{1}{l}{}                   & \multicolumn{1}{l}{}                                    & \multicolumn{1}{l}{}                                        & \multicolumn{1}{l}{}                                  \\
Quality               & \textbf{(0.08)}                        & \textbf{(0.35)}                        & \textbf{(0.34)}                        & 0.52                                   & 0.29                                                     & 1.00                                   & \textbf{}                                               & \textbf{}                                                   & \textbf{}                                             \\
                      & {\color[HTML]{303498} \textbf{0.17}}   & {\color[HTML]{303498} \textbf{(0.48)}} & {\color[HTML]{303498} \textbf{(0.25)}} & {\color[HTML]{303498} 0.76}            & {\color[HTML]{303498} 0.71}                              & {\color[HTML]{303498} 1.00}            & \multicolumn{1}{l}{}                                    & \multicolumn{1}{l}{}                                        & \multicolumn{1}{l}{}                                  \\ \midrule
Credit Spread         & 0.48                                   & \textbf{0.16}                          & \textbf{(0.05)}                        & \textbf{(0.14)}                        & \textbf{(0.37)}                                          & \textbf{(0.07)}                        & 1.00                                                    & \textbf{}                                                   & \textbf{}                                             \\
                      & {\color[HTML]{303498} 0.21}            & {\color[HTML]{303498} \textbf{0.02}}   & {\color[HTML]{303498} \textbf{0.01}}   & {\color[HTML]{303498} \textbf{(0.00)}} & {\color[HTML]{303498} \textbf{(0.05)}}                   & {\color[HTML]{303498} \textbf{(0.03)}} & {\color[HTML]{303498} 1.00}                             & \multicolumn{1}{l}{}                                        & \multicolumn{1}{l}{}                                  \\
High Yield Spread     & 0.50                                   & \textbf{0.13}                          & \textbf{0.02}                          & \textbf{(0.15)}                        & \textbf{(0.40)}                                          & \textbf{(0.11)}                        & 0.65                                                    & 1.00                                                        & \textbf{}                                             \\
                      & {\color[HTML]{303498} 0.38}            & {\color[HTML]{303498} \textbf{0.02}}   & {\color[HTML]{303498} \textbf{0.13}}   & {\color[HTML]{303498} \textbf{(0.06)}} & {\color[HTML]{303498} \textbf{(0.25)}}                   & {\color[HTML]{303498} \textbf{(0.09)}} & {\color[HTML]{303498} \textbf{0.08}}                    & {\color[HTML]{303498} 1.00}                                 & \multicolumn{1}{l}{}                                  \\
Term Spread           & \textbf{(0.32)}                        & \textbf{(0.09)}                        & \textbf{0.02}                          & \textbf{0.10}                          & 0.33                                                     & \textbf{0.03}                          & \textbf{(0.60)}                                         & \textbf{(0.65)}                                             & 1.00                                                  \\
                      & {\color[HTML]{303498} \textbf{(0.30)}} & {\color[HTML]{303498} \textbf{(0.03)}} & {\color[HTML]{303498} \textbf{(0.11)}} & {\color[HTML]{303498} \textbf{0.08}}   & {\color[HTML]{303498} 0.25}                              & {\color[HTML]{303498} \textbf{0.07}}   & {\color[HTML]{303498} \textbf{0.06}}                    & {\color[HTML]{303498} \textbf{(0.78)}}                      & {\color[HTML]{303498} 1.00}                           \\ \midrule
MSCI EAFE             & 0.78                                   & \textbf{0.12}                          & \textbf{(0.19)}                        & \textbf{(0.10)}                        & \textbf{(0.46)}                                          & \textbf{(0.16)}                        & 0.41                                                    & 0.45                                                        & \textbf{(0.22)}                                       \\
                      & {\color[HTML]{303498} 0.49}            & {\color[HTML]{303498} \textbf{0.02}}   & {\color[HTML]{303498} \textbf{0.11}}   & {\color[HTML]{303498} \textbf{0.02}}   & {\color[HTML]{303498} \textbf{(0.16)}}                   & {\color[HTML]{303498} \textbf{(0.03)}} & {\color[HTML]{303498} 0.34}                             & {\color[HTML]{303498} 0.39}                                 & {\color[HTML]{303498} \textbf{(0.23)}}                \\
MSCI JAPAN            & 0.35                                   & \textbf{0.04}                          & \textbf{(0.04)}                        & \textbf{(0.03)}                        & \textbf{(0.14)}                                          & \textbf{(0.26)}                        & 0.22                                                    & 0.27                                                        & \textbf{(0.09)}                                       \\
                      & {\color[HTML]{303498} \textbf{0.14}}   & {\color[HTML]{303498} \textbf{0.00}}   & {\color[HTML]{303498} \textbf{0.02}}   & {\color[HTML]{303498} \textbf{(0.01)}} & {\color[HTML]{303498} \textbf{(0.07)}}                   & {\color[HTML]{303498} \textbf{(0.03)}} & {\color[HTML]{303498} 0.24}                             & {\color[HTML]{303498} 0.21}                                 & {\color[HTML]{303498} \textbf{(0.09)}}                \\
MSCI USA              & 0.87                                   & \textbf{0.00}                          & \textbf{(0.21)}                        & \textbf{0.09}                          & \textbf{(0.50)}                                          & 0.26                                   & 0.40                                                    & 0.38                                                        & \textbf{(0.29)}                                       \\
                      & {\color[HTML]{303498} 0.99}            & {\color[HTML]{303498} \textbf{(0.13)}} & {\color[HTML]{303498} \textbf{0.12}}   & {\color[HTML]{303498} 0.23}            & {\color[HTML]{303498} \textbf{(0.12)}}                   & {\color[HTML]{303498} \textbf{0.20}}   & {\color[HTML]{303498} 0.21}                             & {\color[HTML]{303498} 0.37}                                 & {\color[HTML]{303498} \textbf{(0.30)}}                \\
MSCI EM               & 0.64                                   & \textbf{0.02}                          & \textbf{(0.06)}                        & \textbf{0.00}                          & \textbf{(0.40)}                                          & \textbf{(0.08)}                        & 0.42                                                    & 0.45                                                        & \textbf{(0.20)}                                       \\
                      & {\color[HTML]{303498} 0.46}            & {\color[HTML]{303498} \textbf{(0.04)}} & {\color[HTML]{303498} \textbf{0.11}}   & {\color[HTML]{303498} \textbf{0.10}}   & {\color[HTML]{303498} \textbf{(0.13)}}                   & {\color[HTML]{303498} \textbf{0.03}}   & {\color[HTML]{303498} 0.33}                             & {\color[HTML]{303498} 0.39}                                 & {\color[HTML]{303498} \textbf{(0.21)}}                \\ \midrule
Citi World Bond Index & \textbf{(0.02)}                        & \textbf{(0.02)}                        & \textbf{0.10}                          & \textbf{0.00}                          & \textbf{0.16}                                            & \textbf{(0.10)}                        & \textbf{(0.26)}                                         & \textbf{(0.30)}                                             & 0.40                                                  \\
                      & {\color[HTML]{303498} \textbf{(0.14)}} & {\color[HTML]{303498} \textbf{0.03}}   & {\color[HTML]{303498} \textbf{(0.03)}} & {\color[HTML]{303498} \textbf{0.01}}   & {\color[HTML]{303498} \textbf{0.09}}                     & {\color[HTML]{303498} \textbf{(0.02)}} & {\color[HTML]{303498} \textbf{0.03}}                    & {\color[HTML]{303498} \textbf{(0.35)}}                      & {\color[HTML]{303498} 0.38}                           \\ \bottomrule
\end{tabular}}
\medskip
\caption*{\textit{Notes}: Bold denotes values lower than or equal to 0.20. Dataset A values in black, Dataset B values in blue.}
\end{table}

Confirming results from \citeA{bender2010portfolio}, overall the style-related strategies seem to offer a more robust risk-return profile with respect to the asset class exposures, especially if focusing on Dataset A, i.e. not exclusively on a US space. However, several differences must be highlighted. Firstly, one of the negative values in Table \ref{1997_2020_RR_table} relates to the geometric average return of the valuation factor (and, as a consequence, its annualized sharpe ratio will be negative). Growth stocks are those that grow significantly above the average growth rate of the market, offering strong earnings surge and sometimes not paying dividends. Value stocks, instead, are undervalued with respect to their fundamental metrics.\footnote{Nowadays, the technology sector presents a number of growth stocks, whereas financials, energy and industrials groups tend to consist of value stocks. This is coherent with the numbers in Figures \ref{fig:figure6} and \ref{fig:figure7}.} However, growth stocks' P/E have never been higher as in today's environment and the Growth Index has been overwhelming the Value Index, symptom that the value premium is reverting. \citeA{bender2010portfolio} stopped their analysis in September 2009, when value granted a risk-return performance coherent with size and momentum and it was constantly beating growth stocks. Nevertheless, this opposite situation is not new in the industry (Figure \ref{VALUEGROWTH} in the Appendix) as the theme of the value premium reversion has been of great interest during recent years. The scarcity of growth stocks plays a crucial role, as the direct consequence is that the market will prefer to reward them rather than value stocks. Different studies also criticize the usage of P/E when structuring value-exposed indices, because the usage of this metric has become an irrelevant market practice. \citeA{israel2020systematic} at AQR Capital Management recently addressed this last point in one of their articles, wondering whether systematic value investing is dead or not. According to the study, this seems to be premature, as fundamentals will always play a role in determining returns. However, the abundance of share repurchase programs and the growth in importance of intangible assets are questioning the way value factors are built, making it difficult to have a unique answer to whether this represents a short-term behavior or a vigorous long-term attitude that is here to stay.

Another important behavior regards the size factor. Not only its annualized performance is limited with respect to the market, momentum and quality, but when considering the US environment this involves negative sharpe ratio. One way to analyze this pattern is to start from how this long-short portfolio is structured and deepen the composition of these indices. The Russell 2000 Index is a subset of the Russell 3000 representing approximately 8\% of the total market capitalization of that index. Also the Russell 1000 Index is a subset of the Russell 3000, but it includes approximately 1,000 of the largest securities based on a combination of their market cap and current index membership. However, the R1000 assigns more weight to tech stocks than the R2000 does (28.3\% vs. 13.8\%).\footnote{\textit{Source:} Bloomberg as of the July rebalancing of both indices.} The recent tech boom seems to play a more significant role here, almost overwhelming the several years in which small cap stocks dominated large caps in terms of performance. In addition, both indices significantly weigh financial services (\#2 in the R1000 with 17.2\% and \#1 in the R2000 with 22.8\%), helping to justify the negative contribution of this risk premium. It must be said that this trend is only US based, because size guarantees positive performance (small, but at least positive) during the long run, when considering worldwide stocks. Differently from valuation, empirical evidence was hardly skeptical about this factor, sign that this could be only the result of the recent pandemic crisis.

Discussing further results, equities remain the riskiest markets (in terms of Volatility, VaR and ES) both when looking at asset class and style exposures. In the former this is deepened, but more risk is accompanied by greater returns. More interestingly, if excluding the broad US market (i.e. MSCI USA), styles ensure better performance than asset class exposure, with the exception of the low volatility factor (and, as discussed, valuation).   

In addition, styles offer greater diversification effect, with most of the values in Table \ref{1997_2020_corr_table} being not only low if positive, but also frequently negative.\footnote{Note that the correlation tables throughout the whole work voluntarily do not show the correlations among the asset class exposures themselves, as not of particular interest.}

\subsection{Financial distress periods taken individually}
This section intends to study the risk-return profile of the first five selected periods, as an independent section is dedicated for the COVID-19 pandemic. Figures \ref{fig:figure8}, \ref{fig:figure9}, \ref{fig:figure10} and Table \ref{table:SR_BY_CRISIS} show a summary of the analytical findings, which are then discussed by single crisis. One general comment is that both Datasets show coherence between each other, with the US situation generally being either a worse or slightly better case than when considering a broader geographical universe of stocks. \textit{Considering returns only}, with the exceptions of the market and momentum factors (the last one being extremely related to the first one, as if the US stock market performs better than the worldwide stock market, likely so will the best US companies vs. the best worldwide companies), Dataset A guarantees more positive or less negative returns, regardless of the crisis we consider. Regarding the fixed-income environment, this is not the case for the term spread, but numbers are almost identical between the two datasets. The situation becomes more heterogeneous when involving risk and this is why the work was subdivided crisis-by-crisis, crossing results by factor when needed. Furthermore, despite being of compelling interest, the risk-return performance of the style factors vs. asset classes does not show the complete picture. Indeed, fund managers ought to respect diversification objectives that allow them to avoid "putting all the eggs in the same basket", especially during financial distress periods. Therefore, Table \ref{table:CORR_BY_CRISIS_A} investigates the potential diversification effect, exactly as done for the 1997-2020 interval.\\

%RETURN
\begin{figure}[H]
\centering
\begin{minipage}[b]{\textwidth}
\centering
\captionof{figure}{Annualized returns of asset classes and style factors by crisis.}
\includegraphics[scale=0.6]{RETURNS_BY_CRISIS.pdf}
\caption*{\textit{Notes}: A. or B. in front of the factor indicates the respective Dataset.}
\label{fig:figure8}
\end{minipage}%
\end{figure}

%VOL
\begin{figure}[H]
\centering
\begin{minipage}[b]{\textwidth}
\centering
\captionof{figure}{Annualized volatility of asset classes and style factors by crisis.}
\includegraphics[scale=0.6]{VOL_BY_CRISIS.pdf}
\caption*{\textit{Notes}: A. or B. in front of the factor indicates the respective Dataset.}
\label{fig:figure9}
\end{minipage}%
\end{figure}

%SR
\begin{table}[H]
\caption{Sharpe Ratios of Datasets A and B's factors by crisis.}
\label{table:SR_BY_CRISIS}
\resizebox{\textwidth}{!}{%
\begin{tabular}{@{}llccccc@{}}
\toprule
\textit{Exposure}             & Risk premium      & \multicolumn{1}{l}{Asian Crisis 1997} & \multicolumn{1}{l}{August 1998} & \multicolumn{1}{l}{Dot-com + 11Sept} & \multicolumn{1}{l}{2008 Crisis} & \multicolumn{1}{l}{Eur. Debt Crisis} \\ \midrule
\textit{Asset Class (Equity)} & MSCI EAFE         & 0.50                                  & 0.45                            & (0.94)                               & (0.62)                          & 0.02                                 \\
                              & MSCI JAPAN        & (0.53)                                & 0.23                            & (0.90)                               & (0.73)                          & (0.11)                               \\
                              & MSCI USA          & 1.56                                  & 0.61                            & (0.67)                               & (0.49)                          & 0.46                                 \\
                              & MSCI EM           & (0.94)                                & 0.53                            & (0.91)                               & (0.48)                          & 0.11                                 \\ \midrule
\textit{Asset Class (F.I.)}   & Citi World Bond   & 1.30                                  & 0.41                            & 1.03                                 & 0.93                            & 0.73                                 \\ \midrule
\textit{Style (Equity)}       & Market            & 3.18                                  & 0.53                            & (1.03)                               & (0.61)                          & 0.38                                 \\
                              &                   & {\color[HTML]{303498} 1.16}           & {\color[HTML]{303498} 0.42}     & {\color[HTML]{303498} (0.69)}        & {\color[HTML]{303498} (0.47)}   & {\color[HTML]{303498} 0.56}          \\
                              & Valuation         & (4.68)                                & (0.74)                          & 0.97                                 & (0.18)                          & (0.45)                               \\
                              &                   & {\color[HTML]{303498} (0.83)}         & {\color[HTML]{303498} (1.07)}   & {\color[HTML]{303498} 1.00}          & {\color[HTML]{303498} (0.31)}   & {\color[HTML]{303498} (0.08)}        \\
                              & Size              & (2.05)                                & (0.23)                          & 0.88                                 & 0.17                            & 0.68                                 \\
                              &                   & {\color[HTML]{303498} (1.53)}         & {\color[HTML]{303498} (0.31)}   & {\color[HTML]{303498} 0.50}          & {\color[HTML]{303498} 0.10}     & {\color[HTML]{303498} 0.22}          \\
                              & Momentum          & 0.62                                  & 0.43                            & (0.22)                               & (0.54)                          & 0.64                                 \\
                              &                   & {\color[HTML]{303498} 0.37}           & {\color[HTML]{303498} 0.35}     & {\color[HTML]{303498} 0.17}          & {\color[HTML]{303498} (0.48)}   & {\color[HTML]{303498} 0.37}          \\
                              & Low Vol           & (0.18)                                & (0.64)                          & 1.11                                 & 0.25                            & 0.09                                 \\
                              &                   & {\color[HTML]{303498} (0.25)}         & {\color[HTML]{303498} (0.20)}   & {\color[HTML]{303498} 0.79}          & {\color[HTML]{303498} 0.30}     & {\color[HTML]{303498} 0.07}          \\
                              & Quality           & 0.69                                  & 0.34                            & 0.64                                 & 0.99                            & 0.42                                 \\
                              &                   & {\color[HTML]{303498} 0.27}           & {\color[HTML]{303498} 0.12}     & {\color[HTML]{303498} 0.26}          & {\color[HTML]{303498} 0.81}     & {\color[HTML]{303498} 0.04}          \\ \midrule
\textit{Style (F.I.)}         & Credit Spread     & (0.83)                                & (0.44)                          & (0.24)                               & 0.09                            & 0.23                                 \\
                              &                   & {\color[HTML]{303498} (0.13)}         & {\color[HTML]{303498} (0.52)}   & {\color[HTML]{303498} (0.28)}        & {\color[HTML]{303498} (0.16)}   & {\color[HTML]{303498} 1.36}          \\
                              & High Yield Spread & (0.24)                                & (1.01)                          & (1.60)                               & (0.29)                          & 0.87                                 \\
                              &                   & {\color[HTML]{303498} (0.45)}         & {\color[HTML]{303498} (0.76)}   & {\color[HTML]{303498} (1.61)}        & {\color[HTML]{303498} (0.14)}   & {\color[HTML]{303498} 0.36}          \\
                              & Term Spread       & 0.98                                  & (0.08)                          & 0.68                                 & 0.24                            & 0.87                                 \\
                              &                   & {\color[HTML]{303498} 0.99}           & {\color[HTML]{303498} (0.05)}   & {\color[HTML]{303498} 0.60}          & {\color[HTML]{303498} 0.15}     & {\color[HTML]{303498} 0.81}          \\ \bottomrule
\end{tabular}}
\medskip
\caption*{\textit{Notes}: Dataset A values in black, Dataset B values in blue.}
\end{table}

%VAR_ES
\begin{figure}[H]
\centering
\begin{minipage}[b]{\textwidth}
\centering
\captionof{figure}{Historical 95\% VaR and ES of asset classes and style factors by crisis (VaR left in negative terms, i.e. bottom half, whereas ES in absolute value, i.e. top half).}
\includegraphics[scale=0.58]{VAR_ES_BY_CRISIS.pdf}
\bigskip
\caption*{\textit{Notes}: Parametric 95\% VaR not included due to space reasons (historical VaR was empirically greater in absolute value in both datasets, representing an even more conservative estimate).}
\label{fig:figure10}
\end{minipage}
\end{figure}

% Please add the following required packages to your document preamble:
% \usepackage{booktabs}
\begin{table}[H]
\caption{Dataset A - Correlation Table by crisis.}
\label{table:CORR_BY_CRISIS_A}
\resizebox{\textwidth}{!}{%
\begin{tabular}{@{}llcccccccccccccc@{}}
\toprule
                  &                   & Market          & Valuation       & Size            & Momentum        & Low Vol         & Quality         & Credit Spread   & H.Y. Spread     & Term Spread     & MSCI EAFE       & MSCI JAPAN      & MSCI USA        & MSCI EM         & Citi World Bond \\ \midrule
Market            & Asian Crisis 1997 & 1.00            & \textbf{}       & \textbf{}       & \textbf{}       & \textbf{}       & \textbf{}       & \textbf{}       & \textbf{}       &                 &                 &                 &                 &                 &                 \\
                  & August 1998       & 1.00            &                 &                 &                 &                 &                 &                 &                 &                 &                 &                 &                 &                 &                 \\
                  & Dot-com + 11Sept  & 1.00            &                 &                 &                 &                 &                 &                 &                 &                 &                 &                 &                 &                 &                 \\
                  & 2008 Crisis       & 1.00            &                 &                 &                 &                 &                 &                 &                 &                 &                 &                 &                 &                 &                 \\
                  & Eur. Debt Crisis  & 1.00            &                 &                 &                 &                 &                 &                 &                 &                 &                 &                 &                 &                 &                 \\
Valuation         & Asian Crisis 1997 & \textbf{(0.00)} & 1.00            & \textbf{}       & \textbf{}       & \textbf{}       & \textbf{}       & \textbf{}       & \textbf{}       &                 &                 &                 &                 &                 &                 \\
                  & August 1998       & \textbf{(0.46)} & 1.00            &                 &                 &                 &                 &                 &                 &                 &                 &                 &                 &                 &                 \\
                  & Dot-com + 11Sept  & \textbf{(0.20)} & 1.00            &                 &                 &                 &                 &                 &                 &                 &                 &                 &                 &                 &                 \\
                  & 2008 Crisis       & 0.48            & 1.00            &                 &                 &                 &                 &                 &                 &                 &                 &                 &                 &                 &                 \\
                  & Eur. Debt Crisis  & \textbf{0.09}   & 1.00            &                 &                 &                 &                 &                 &                 &                 &                 &                 &                 &                 &                 \\
Size              & Asian Crisis 1997 & \textbf{(0.03)} & \textbf{0.10}   & 1.00            & \textbf{}       & \textbf{}       & \textbf{}       & \textbf{}       & \textbf{}       &                 &                 &                 &                 &                 &                 \\
                  & August 1998       & \textbf{(0.52)} & 0.22            & 1.00            &                 &                 &                 &                 &                 &                 &                 &                 &                 &                 &                 \\
                  & Dot-com + 11Sept  & \textbf{(0.61)} & \textbf{0.12}   & 1.00            &                 &                 &                 &                 &                 &                 &                 &                 &                 &                 &                 \\
                  & 2008 Crisis       & \textbf{(0.19)} & \textbf{(0.11)} & 1.00            &                 &                 &                 &                 &                 &                 &                 &                 &                 &                 &                 \\
                  & Eur. Debt Crisis  & 0.24            & \textbf{(0.29)} & 1.00            &                 &                 &                 &                 &                 &                 &                 &                 &                 &                 &                 \\
Momentum          & Asian Crisis 1997 & \textbf{(0.01)} & \textbf{0.00}   & \textbf{(0.67)} & 1.00            & \textbf{}       & \textbf{}       & \textbf{}       & \textbf{}       &                 &                 &                 &                 &                 &                 \\
                  & August 1998       & 0.32            & \textbf{(0.55)} & \textbf{(0.37)} & 1.00            &                 &                 &                 &                 &                 &                 &                 &                 &                 &                 \\
                  & Dot-com + 11Sept  & \textbf{(0.31)} & \textbf{(0.33)} & 0.32            & 1.00            &                 &                 &                 &                 &                 &                 &                 &                 &                 &                 \\
                  & 2008 Crisis       & \textbf{(0.38)} & \textbf{(0.43)} & \textbf{0.18}   & 1.00            &                 &                 &                 &                 &                 &                 &                 &                 &                 &                 \\
                  & Eur. Debt Crisis  & \textbf{(0.20)} & \textbf{(0.39)} & 0.25            & 1.00            &                 &                 &                 &                 &                 &                 &                 &                 &                 &                 \\
Low Vol           & Asian Crisis 1997 & \textbf{(0.15)} & \textbf{0.02}   & \textbf{(0.60)} & 0.87            & 1.00            & \textbf{}       & \textbf{}       & \textbf{}       &                 &                 &                 &                 &                 &                 \\
                  & August 1998       & \textbf{(0.49)} & 0.45            & \textbf{0.00}   & \textbf{0.18}   & 1.00            &                 &                 &                 &                 &                 &                 &                 &                 &                 \\
                  & Dot-com + 11Sept  & \textbf{(0.80)} & 0.41            & 0.48            & 0.29            & 1.00            &                 &                 &                 &                 &                 &                 &                 &                 &                 \\
                  & 2008 Crisis       & \textbf{(0.83)} & \textbf{(0.30)} & 0.22            & 0.35            & 1.00            &                 &                 &                 &                 &                 &                 &                 &                 &                 \\
                  & Eur. Debt Crisis  & \textbf{(0.92)} & \textbf{(0.09)} & \textbf{(0.20)} & 0.30            & 1.00            &                 &                 &                 &                 &                 &                 &                 &                 &                 \\
Quality           & Asian Crisis 1997 & \textbf{(0.03)} & \textbf{(0.01)} & \textbf{(0.69)} & 0.88            & 0.75            & 1.00            & \textbf{}       & \textbf{}       &                 &                 &                 &                 &                 &                 \\
                  & August 1998       & \textbf{0.18}   & \textbf{(0.31)} & \textbf{(0.46)} & 0.73            & 0.32            & 1.00            &                 &                 &                 &                 &                 &                 &                 &                 \\
                  & Dot-com + 11Sept  & \textbf{0.02}   & \textbf{(0.53)} & \textbf{(0.19)} & \textbf{0.19}   & \textbf{(0.07)} & 1.00            &                 &                 &                 &                 &                 &                 &                 &                 \\
                  & 2008 Crisis       & \textbf{(0.23)} & \textbf{(0.54)} & \textbf{(0.16)} & 0.29            & \textbf{0.08}   & 1.00            &                 &                 &                 &                 &                 &                 &                 &                 \\
                  & Eur. Debt Crisis  & \textbf{(0.51)} & \textbf{(0.41)} & \textbf{(0.08)} & 0.54            & 0.58            & 1.00            &                 &                 &                 &                 &                 &                 &                 &                 \\
Credit Spread     & Asian Crisis 1997 & \textbf{0.18}   & \textbf{0.11}   & \textbf{(0.17)} & 0.31            & 0.21            & 0.31            & 1.00            & \textbf{}       &                 &                 &                 &                 &                 &                 \\
                  & August 1998       & \textbf{0.10}   & \textbf{(0.05)} & \textbf{(0.15)} & 0.25            & \textbf{0.10}   & 0.30            & 1.00            &                 &                 &                 &                 &                 &                 &                 \\
                  & Dot-com + 11Sept  & 0.34            & \textbf{0.03}   & \textbf{(0.20)} & \textbf{(0.28)} & \textbf{(0.33)} & \textbf{(0.09)} & 1.00            &                 &                 &                 &                 &                 &                 &                 \\
                  & 2008 Crisis       & 0.53            & 0.35            & \textbf{(0.12)} & \textbf{(0.26)} & \textbf{(0.48)} & \textbf{(0.27)} & 1.00            &                 &                 &                 &                 &                 &                 &                 \\
                  & Eur. Debt Crisis  & 0.65            & \textbf{0.16}   & \textbf{0.17}   & \textbf{(0.22)} & \textbf{(0.66)} & \textbf{(0.40)} & 1.00            &                 &                 &                 &                 &                 &                 &                 \\
High Yield Spread & Asian Crisis 1997 & 0.33            & \textbf{0.15}   & \textbf{0.05}   & \textbf{0.00}   & \textbf{(0.04)} & \textbf{0.01}   & 0.62            & 1.00            &                 &                 &                 &                 &                 &                 \\
                  & August 1998       & \textbf{(0.02)} & \textbf{0.02}   & \textbf{0.06}   & \textbf{0.08}   & \textbf{0.07}   & \textbf{0.09}   & 0.44            & 1.00            &                 &                 &                 &                 &                 &                 \\
                  & Dot-com + 11Sept  & 0.27            & \textbf{0.07}   & \textbf{(0.07)} & \textbf{(0.20)} & \textbf{(0.27)} & \textbf{(0.14)} & 0.55            & 1.00            &                 &                 &                 &                 &                 &                 \\
                  & 2008 Crisis       & 0.53            & 0.23            & \textbf{(0.09)} & \textbf{(0.30)} & \textbf{(0.46)} & \textbf{(0.27)} & 0.58            & 1.00            &                 &                 &                 &                 &                 &                 \\
                  & Eur. Debt Crisis  & 0.66            & \textbf{0.13}   & \textbf{0.12}   & \textbf{(0.22)} & \textbf{(0.67)} & \textbf{(0.40)} & 0.78            & 1.00            &                 &                 &                 &                 &                 &                 \\
Term Spread       & Asian Crisis 1997 & \textbf{(0.03)} & \textbf{(0.06)} & \textbf{(0.04)} & \textbf{(0.01)} & \textbf{(0.01)} & \textbf{(0.03)} & \textbf{(0.66)} & \textbf{(0.86)} &                 &                 &                 &                 &                 &                 \\
                  & August 1998       & \textbf{0.11}   & \textbf{(0.03)} & \textbf{(0.01)} & \textbf{(0.08)} & \textbf{(0.15)} & \textbf{(0.15)} & \textbf{(0.60)} & \textbf{(0.86)} & 1.00            &                 &                 &                 &                 &                 \\
                  & Dot-com + 11Sept  & \textbf{(0.26)} & \textbf{(0.04)} & \textbf{0.18}   & 0.21            & 0.25            & \textbf{0.04}   & \textbf{(0.80)} & \textbf{(0.60)} & 1.00            &                 &                 &                 &                 &                 \\
                  & 2008 Crisis       & \textbf{(0.33)} & \textbf{(0.18)} & \textbf{(0.01)} & \textbf{0.09}   & 0.35            & \textbf{0.06}   & \textbf{(0.58)} & \textbf{(0.53)} & 1.00            &                 &                 &                 &                 &                 \\
                  & Eur. Debt Crisis  & \textbf{(0.58)} & \textbf{(0.16)} & \textbf{(0.10)} & \textbf{0.16}   & 0.58            & 0.26            & \textbf{(0.79)} & \textbf{(0.78)} & 1.00            &                 &                 &                 &                 &                 \\
MSCI EAFE         & Asian Crisis 1997 & 0.21            & \textbf{(0.00)} & \textbf{(0.39)} & 0.68            & 0.74            & 0.41            & 0.25            & \textbf{0.16}   & \textbf{(0.13)} & 1.00            &                 &                 &                 &                 \\
                  & August 1998       & 0.55            & \textbf{(0.21)} & \textbf{(0.31)} & 0.53            & \textbf{0.15}   & \textbf{0.16}   & \textbf{0.17}   & \textbf{0.10}   & \textbf{(0.05)} & 1.00            &                 &                 &                 &                 \\
                  & Dot-com + 11Sept  & 0.77            & \textbf{(0.04)} & \textbf{(0.45)} & \textbf{(0.28)} & \textbf{(0.56)} & \textbf{(0.35)} & 0.26            & 0.23            & \textbf{(0.17)} & 1.00            &                 &                 &                 &                 \\
                  & 2008 Crisis       & 0.83            & 0.36            & \textbf{(0.38)} & \textbf{(0.53)} & \textbf{(0.77)} & \textbf{(0.30)} & 0.49            & 0.54            & \textbf{(0.22)} & 1.00            &                 &                 &                 &                 \\
                  & Eur. Debt Crisis  & 0.90            & \textbf{0.12}   & \textbf{0.04}   & \textbf{(0.43)} & \textbf{(0.86)} & \textbf{(0.67)} & 0.59            & 0.62            & \textbf{(0.47)} & 1.00            &                 &                 &                 &                 \\
MSCI JAPAN        & Asian Crisis 1997 & \textbf{0.13}   & \textbf{(0.02)} & \textbf{(0.20)} & 0.30            & 0.44            & \textbf{0.13}   & \textbf{0.09}   & \textbf{0.09}   & \textbf{(0.05)} & 0.68            & 1.00            &                 &                 &                 \\
                  & August 1998       & 0.28            & \textbf{(0.05)} & \textbf{(0.10)} & 0.21            & \textbf{0.12}   & \textbf{(0.07)} & \textbf{0.04}   & \textbf{0.06}   & \textbf{(0.01)} & 0.61            & 1.00            &                 &                 &                 \\
                  & Dot-com + 11Sept  & 0.33            & \textbf{0.01}   & \textbf{(0.09)} & \textbf{(0.07)} & \textbf{(0.19)} & \textbf{(0.45)} & \textbf{0.11}   & \textbf{0.13}   & \textbf{(0.08)} & 0.49            & 1.00            &                 &                 &                 \\
                  & 2008 Crisis       & 0.42            & \textbf{0.16}   & \textbf{(0.23)} & \textbf{(0.39)} & \textbf{(0.27)} & \textbf{(0.52)} & 0.33            & 0.44            & \textbf{(0.08)} & 0.66            & 1.00            &                 &                 &                 \\
                  & Eur. Debt Crisis  & 0.36            & \textbf{(0.03)} & \textbf{0.12}   & \textbf{(0.31)} & \textbf{(0.36)} & \textbf{(0.51)} & 0.28            & 0.30            & \textbf{(0.15)} & 0.50            & 1.00            &                 &                 &                 \\
MSCI USA          & Asian Crisis 1997 & \textbf{0.19}   & \textbf{0.01}   & \textbf{(0.68)} & 0.81            & 0.67            & 0.90            & 0.32            & \textbf{0.04}   & \textbf{0.01}   & 0.31            & \textbf{0.10}   & 1.00            &                 &                 \\
                  & August 1998       & 0.73            & \textbf{(0.37)} & \textbf{(0.65)} & 0.58            & \textbf{(0.11)} & 0.72            & 0.22            & \textbf{0.01}   & \textbf{0.01}   & 0.28            & \textbf{0.07}   & 1.00            &                 &                 \\
                  & Dot-com + 11Sept  & 0.91            & \textbf{(0.25)} & \textbf{(0.59)} & \textbf{(0.26)} & \textbf{(0.76)} & 0.27            & 0.31            & 0.23            & \textbf{(0.25)} & 0.45            & \textbf{0.14}   & 1.00            &                 &                 \\
                  & 2008 Crisis       & 0.88            & 0.46            & \textbf{0.02}   & \textbf{(0.16)} & \textbf{(0.66)} & \textbf{(0.11)} & 0.41            & 0.36            & \textbf{(0.33)} & 0.47            & \textbf{0.12}   & 1.00            &                 &                 \\
                  & Eur. Debt Crisis  & 0.91            & \textbf{0.06}   & 0.36            & \textbf{0.05}   & \textbf{(0.80)} & \textbf{(0.27)} & 0.60            & 0.57            & \textbf{(0.59)} & 0.65            & \textbf{0.16}   & 1.00            &                 &                 \\
MSCI EM           & Asian Crisis 1997 & 0.29            & \textbf{0.06}   & \textbf{(0.31)} & 0.58            & 0.55            & 0.46            & 0.37            & 0.31            & \textbf{(0.22)} & 0.63            & 0.37            & 0.48            & 1.00            &                 \\
                  & August 1998       & 0.47            & \textbf{(0.21)} & \textbf{(0.17)} & 0.47            & \textbf{0.00}   & 0.24            & 0.24            & 0.21            & \textbf{(0.11)} & 0.63            & 0.42            & 0.39            & 1.00            &                 \\
                  & Dot-com + 11Sept  & 0.53            & \textbf{(0.21)} & \textbf{(0.12)} & \textbf{(0.13)} & \textbf{(0.52)} & \textbf{(0.22)} & 0.22            & 0.28            & \textbf{(0.08)} & 0.59            & 0.47            & 0.37            & 1.00            &                 \\
                  & 2008 Crisis       & 0.77            & 0.28            & \textbf{(0.21)} & \textbf{(0.37)} & \textbf{(0.70)} & \textbf{(0.33)} & 0.53            & 0.54            & \textbf{(0.26)} & 0.85            & 0.66            & 0.51            & 1.00            &                 \\
                  & Eur. Debt Crisis  & 0.74            & \textbf{(0.05)} & \textbf{0.16}   & \textbf{(0.31)} & \textbf{(0.76)} & \textbf{(0.57)} & 0.56            & 0.61            & \textbf{(0.40)} & 0.82            & 0.54            & 0.53            & 1.00            &                 \\
Citi World Bond   & Asian Crisis 1997 & \textbf{(0.05)} & \textbf{0.00}   & \textbf{0.11}   & \textbf{(0.06)} & \textbf{0.14}   & \textbf{(0.14)} & \textbf{(0.25)} & \textbf{(0.22)} & \textbf{0.17}   & \textbf{0.13}   & \textbf{0.01}   & \textbf{(0.15)} & \textbf{(0.12)} & 1.00            \\
                  & August 1998       & \textbf{0.00}   & \textbf{0.04}   & \textbf{0.11}   & \textbf{(0.18)} & \textbf{0.11}   & \textbf{(0.23)} & \textbf{(0.23)} & \textbf{(0.30)} & 0.27            & \textbf{0.08}   & \textbf{(0.07)} & \textbf{(0.15)} & \textbf{(0.17)} & 1.00            \\
                  & Dot-com + 11Sept  & \textbf{(0.15)} & \textbf{(0.00)} & 0.20            & \textbf{0.15}   & 0.31            & \textbf{(0.04)} & \textbf{(0.37)} & \textbf{(0.37)} & 0.41            & \textbf{(0.01)} & \textbf{(0.09)} & \textbf{(0.21)} & \textbf{(0.17)} & 1.00            \\
                  & 2008 Crisis       & \textbf{(0.05)} & \textbf{(0.12)} & \textbf{0.01}   & \textbf{(0.03)} & \textbf{0.08}   & \textbf{(0.00)} & \textbf{(0.27)} & \textbf{(0.28)} & 0.42            & \textbf{0.12}   & \textbf{(0.07)} & \textbf{(0.19)} & \textbf{(0.06)} & 1.00            \\
                  & Eur. Debt Crisis  & 0.21            & \textbf{0.07}   & \textbf{(0.11)} & \textbf{(0.30)} & \textbf{(0.12)} & \textbf{(0.30)} & \textbf{(0.11)} & \textbf{(0.14)} & 0.21            & 0.35            & \textbf{(0.08)} & \textbf{0.04}   & \textbf{0.17}   & 1.00            \\ \bottomrule
\end{tabular}}
\medskip
\caption*{\textit{Notes}: Bold denotes values lower than or equal to 0.20.}
\end{table}

\textit{\textbf{Asian Crisis of 1997}}. In terms of market and momentum, the Asian Crisis plays the role of the best-among-the-worst, maybe a symptom that the aftermaths of the crisis were not as interconnected as the other peer timespans. Quality did not behave differently, but before being too optimistic, one consideration has to be done both about quality and momentum. Despite showing acceptable performances, their VaR and ES were the greatest, not only when considering cross-factors within this debacle, but also cross-crisis. When losses occurred, they could have been in the average size of 2.5\% and 2.3\% (respectively, momentum and quality's 95\% ES for the Dataset A; even more amplified numbers were registered in the US space, as it is possible to appreciate in Figure \ref{fig:figure10}).

However, the two crucial outcomes concern the exceptions to the best-among-the-worst instance, i.e. size and valuation. Regarding the former, the 97-98 period hit small companies more than large ones, hence reverting the factor itself. One driver could be the international investors' reluctancy to lend to developing countries (indeed explaining the bad numbers of the MSCI Emerging Markets and MSCI Japan as well), causing more disruptive economic slowdowns in small realities rather than in the already established ones: small cap companies tend to be very proactive in capital markets, often looking for investments to expand their business and the right moment to do it. 

Value stocks suffered a lot during these 2 years. As anticipated, "most explanations of the 1997 East Asian crisis focus on either weaknesses in Asian financial structures (the so-called ‘crony capitalism-moral hazard’ explanation) or on imprudent lending and panicked over-reaction by foreign banks"\cite{democratizing_governance}, but nothing of this directly helps to explain this particular trend. Unfortunately, another issue was raised in terms of data retrievability, as it is not possible to singularly analyze the companies in the MSCI indices without permission from the entity. It is possible to study the behavior of the top 10 constituents of both the MSCI Value Index and MSCI Growth Index, but the underlying assumptions would be that these were top 10 members also during the Asian Crisis, an assumption that does not hold as unrealistic in the first place.
The composition of both indices plays a crucial role here, as the Thai Baht collapse of July 97 is not likely to have interfered with the performances of most of the biggest companies in the indices (regardless of whether they were value or growth), hence justifying the annualized limited volatility of the valuation factor (only c.1.5\%), and finally its absurd sharpe ratio.

In addition, these two indices were the ones missing more 97-98 data, i.e. the ones for which most of the linear interpolation had to be done. Despite being an adequate solution, this inevitably involves a sort of bias that could help explaining the outlier in terms of absolute value in Table \ref{table:SR_BY_CRISIS}.

Finally, it can be appreciated from Table \ref{table:CORR_BY_CRISIS_A} that, when abstracting from returns and focusing on diversification only, styles offer better properties than asset classes. Indeed, size is very negatively correlated with most of the factors, while momentum is so only with the aforementioned size, introducing doubts about the effectiveness of including it in a portfolio which already bets on low volatility and quality stocks. \\

\textit{\textbf{August 1998 turmoil}}. Asset classes seem to have performed better than style factors during the August '98 turmoil. If not considering the market, momentum and quality, both equity and fixed-income oriented factors demonstrated negative annualized returns, hence negative sharpe ratios. However, the same reasoning done for the Asian crisis holds here as well: momentum and quality were among the riskiest factors in terms of VaR and ES. In general, however, the late 1990s collapse of the hedge fund Long-Term Capital Management (LTCM), its successive recapitalization and its conclusive liquidation hit more factor investing rather than "owning the market" by investing in geographical broader indices. This introduces a crucial point to the analysis, i.e. the importance of the nature of the considered crisis. As stated in a recent article from the AQR Capital Management's co-founder and factor-investing guru Clifford Asness, LTCM's strategies were (a) very market directional and (b) most of them focused on fixed-income arbitrage opportunities and convergence trades \cite{cliff_august1998}, i.e. trades exploiting price misalignments. This crisis was mostly generated by a hedge fund collapse\footnote{In addition to the high leverage sustained by LTCM, two other triggers were indeed the exposure to the 1997 Asian Crisis (as anticipated in \textit{Section \ref{section3.2} - \nameref{section3.2}}) and to the 1998 Russian financial crisis, implicitly considered in the period.}, where long-short portfolios were at the very core of the business.

Furthermore, another fascinating result regards the bottom part of Table \ref{table:SR_BY_CRISIS}, i.e. the fixed-income world and the term spread in particular. With respect to the other crises, the duration times spread ensured negative performances. Volatility was coherent with the Asian and Dot-com Crisis, but very limited if comparing with the two more recent timespans. As in most of the explanations behind the F.I. environment, the yield curve helps us understanding why. Indeed, "the yield curve was inverted by .02\% on September 11, 1998, by .12\% on September 21, 1998, by .08\% on September 22, 1998 and by .07\% on October 7, 1998. The Fed intervened again on September 28, 1998 by lowering interest rates 0.25\% and a new bull market in stocks began in early October 1998, continuing until 2000" \cite{RBC_yield_curve1998}. Hence, one explanation to the term spread performance could stand behind the idea that when the Fed and the Central Banks have to manage inflation, they act on short term interest rates (longer maturity interest rates are determined by market forces) and this has happened in a so-tremendous-way that the yield curve was inverted.

On the other hand, when looking at the correlation table, style factors seem to offer robust diversification benefits, as most of the values remain negative or slightly positive. Size offers significant diversification potential with the broad equity market, a pattern that has now been experienced in both the Asian Crisis and the 98 turmoil. \\

\textit{\textbf{Dot-com bubble and 11 Sept}}. The market bubble caused by the over-speculating investments in Internet-related companies sees styles overwhelming broad equity from a risk-return standpoint. Valuation finally guarantees positive return, and this is the only instance among the subdivided periods. As analyzed by \citeA{wheale2003bursting} in a case study on investing behavior during this debacle, few were the companies not over-investing with the mere scope of expansion, because this period is characterized by low levels of profitability in the industry. Inevitably, asset classes were hit regardless of the geography. Growth indices, which as previously mentioned is mostly constituted by tech-related stocks, significantly suffered, generating a positive sharpe ratio for valuation.

In order to further investigate the behavior of the styles factors and how they helped explaining returns during the dot-com bubble, the excess returns of 10 different companies which saw their prices soaring and then going bust (not necessarily filing for bankrupt) were regressed on the equity style factors. Notice how, before running any regression, the Dot-com part of the correlation table had to be confronted in order to avoid multicollinearity, i.e. linear relationships among regressors. In addition, it must be borne in mind that a portfolio diversification angle (that is, the more negative the correlation, the greater the benefits) does not fit in this instance. Instead, the ideal situation would be correlations to be tending to zero, regardless of the sign. Despite this is not the case pre-regressions, the variance inflation factor (i.e. metric used to measure the severity of multicollinearity post OLS regression) is always well below the common cutoff point of 10 for all the regressors (the same holds for a more strict maximum threshold of 5, as it can be seen from Table \ref{VIF_dot-com} in the appendix). \\

\begin{table}[H]
\caption{Results of 10 Tech stocks' excess returns regressions on style factors during the Dot-com Bubble.}
\label{table:DOT_COM_REGRESSION}
\resizebox{\textwidth}{!}{%
\begin{tabular}{@{}lcccccccccc@{}}
\toprule
              & 3Com   & Akamai Technologies & Blue Coat Systems & Blucora   & Digex     & Savvis    & Terra Networks & Inktomi   & TIBCO Software & Covad     \\ \midrule
              &           &                    &                 &           &           &           &                &           &                &           \\
Market        & 0.94***   & 1.80***            & 1.29***         & 1.60***   & 1.37**    & 0.347     & 0.65**         & 2.00***   & 1.59***        & 1.09*     \\
              & (0.26)    & (0.39)             & (0.49)          & (0.43)    & (0.54)    & (0.56)    & (0.31)         & (0.45)    & (0.35)         & (0.61)    \\
Valuation     & 0.17      & (1.55)***          & (2.62)***       & (1.32)**  & (1.21)*   & (1.97)*** & (1.49)***      & (1.65)*** & (1.35)***      & -1.21     \\
              & (0.34)    & (0.51)             & (0.64)          & (0.56)    & (0.71)    & (0.73)    & (0.41)         & (0.59)    & (0.46)         & (0.80)    \\
Size          & 1.28***   & 3.40***            & 1.58**          & 4.04***   & 4.42***   & 2.84***   & 0.45           & 4.32***   & 3.09***        & 2.34***   \\
              & (0.38)    & (0.57)             & (0.71)          & (0.63)    & (0.79)    & (0.82)    & (0.46)         & (0.66)    & (0.52)         & (0.89)    \\
Momentum      & 0.253     & -0.19              & -0.13           & -0.07     & 0.85      & (1.21)*   & -0.43          & (1.46)*** & -0.30          & -0.20     \\
              & (0.30)    & (0.45)             & (0.56)          & (0.50)    & (0.63)    & (0.65)    & (0.36)         & (0.52)    & (0.41)         & (0.71)    \\
LowVolatility & (2.33)*** & (5.46)***          & (4.11)***       & (4.53)*** & (4.82)*** & (2.57)**  & (1.85)***      & (5.94)*** & (4.58)***      & (3.54)*** \\
              & (0.53)    & (0.79)             & (0.99)          & (0.87)    & (1.10)    & (1.14)    & (0.63)         & (0.92)    & (0.72)         & (1.23)    \\
Quality       & 1.68***   & 0.13               & (1.73)**        & 1.02      & 0.00      & -1.18     & (2.77)***      & 1.01      & 1.09*          & 0.66      \\
              & (0.47)    & (0.70)             & (0.87)          & (0.77)    & (0.97)    & (1.01)    & (0.56)         & (0.81)    & (0.63)         & (1.09)    \\
Constant      & 0.00      & (0.00)             & 0.00            & 0.00      & 0.00      & 0.00      & (0.00)         & 0.00      & 0.00           & 0.00      \\
              & (0.00)    & (0.00)             & (0.00)          & (0.00)    & (0.00)    & (0.00)    & (0.00)         & (0.00)    & (0.00)         & (0.00)    \\
              &           &                    &                 &           &           &           &                &           &                &           \\
Observations  & 776       & 776                & 776             & 776       & 776       & 776       & 775            & 776       & 776            & 776       \\
R-squared     & 0.18      & 0.35               & 0.21            & 0.24      & 0.15      & 0.06      & 0.15           & 0.33      & 0.33           & 0.09      \\ \bottomrule
\end{tabular}}
\smallskip
\caption*{\textit{Notes}: Standard errors in parentheses: *** p$<$0.01, ** p$<$0.05, * p$<$0.1 .}
\end{table}

Even if considering Savvis and Covad as outliers (R-squared almost 0), low volatility plays a crucial role in explaining the excess returns of these stocks, coherently with what found in Table \ref{table:SR_BY_CRISIS}, as the sharpe ratio reaches the maximum values from the cross-crisis standpoint. As addressed by \citeA{frazzini2014betting}, betting-against-beta (i.e. going long low-beta assets by shorting high-beta assets) has proved to be a source of positive risk-adjusted returns. Coefficients are significant for all the 10 stocks, all indicating that low volatility is indeed a factor exposure explaining part of the suffering excess returns of these stocks during the bubble. Moreover, the companies' excess returns were regressed on the market only, in order to compute their betas (as previously mentioned, in fact, these are not companies that went necessarily bankrupt after the bubble, rather companies whose prices soared because of the Tech rise of the 2000s but were inevitably hit when it bursted. However, most of these companies were involved in spin-off or M\&A processes, making it difficult to retrieve data for raw betas on Bloomberg). As expected, being Tech companies these represented the short leg of the low volatility, hence justifying the negative sign in front of the coefficients.

Almost the same reasoning can be applied for the size factor, as results in Table \ref{table:SR_BY_CRISIS} are coherent with the fact that the average market cap of the selected companies during the timespan is below \$8bn and coefficients are not only statistically significant (excluding Terra Networks), but of course positive (data for betas and market caps can be appreciated in Table \ref{Betas_market_caps_dot_com} of the Appendix). \\

\textit{\textbf{2008 Financial Crisis}}. Very similarly to the Dot-com Bubble, the financial crisis of 2008 gives another confirmation of the over-performance of factor investing with respect to (mere) asset class investing. As the very lefthand side of Figure \ref{fig:figure9} demonstrates, volatility was the main driver in this case, as the swings in prices of the most important equity indices were relatively higher than the ones in the long-short portfolios, regardless of the considered geography. Historical VaR and ES both add evidence to it (Figure \ref{fig:figure10}).

After a period in which value investing was paying off, the factor reverts again during this financial collapse. However, the behavior of momentum is the key result to be highlighted in this context. This factor exhibits negative sharpe ratio, adding evidence to the entity of this crisis, unquestionably the one in which financial markets were hit the most. 

Moreover, the long-short quality portfolio reaches the best risk-return profile among the different crises and the reason can be found in the index on which it is based (i.e. MSCI World Quality for Dataset A and MSCI World Quality for Dataset B). The index aims to capture the performance of quality growth stocks by identifying stocks with high quality scores based on three main fundamental variables: high return on equity (ROE), stable year-over-year earnings growth and low financial leverage\footnote{\textit{Source}: Bloomberg.}. In particular, this last measure used to classify if a company appertains to the Quality index or not is crucial: companies bearing reduced debt-to-equity ratio suffered less the consequences of a distress period in which banks had increased their balance sheet riskiness, struggling to sustain requests for corporate credit (this was extremely amplified in the advanced state of the financial crisis, when housing mortgages where saturating and CDOs on corporate loans were starting to become Wall Street's common practice).

With similar rationale, minimum volatility seems to have resisted to the 2008 financial disaster, as less risky stocks managed to continue their core activities in a more efficient way rather than the market as a whole. In a context where financial frauds and inconsistencies characterized the market, low beta stocks granted greater risk-adjusted performances, with limited average losses when exceeding the 95\% historical VaR.\footnote{The absolute value of expected shortfall for low volatility during 2008 was second to the Debt Crisis only.}

The average pairwise correlation remains lower in the risk-factor world, coherent with \citeA{idzorek2013factor} and holding for all the crises analyzed in the work. Factors have suffered the aftermath of Lehman Brothers' collapse and this definitely represents the worst crisis analyzed since this point of the work. However, consequences were not so extended to revert the diversification benefits, that not only seem to persist, but they also reach their maximum in this particular instance.\footnote{The average of the style factors correlations is negative, lower than the asset class correlations and it is the minimum among the analyzed distress periods.} \\


\textit{\textbf{Peak of European Debt Crisis}}. Before entering into the details of the risk-return conduct of factor investing in this crisis, it is worth remembering its key points and relationship with the 2008's tumults. Often referred as \textit{Eurozone crisis} or \textit{European sovereign debt crisis}, the onset of this debacle dates back to 2009, when Greece announced their deficit had been underestimated and the fiscal situation was worse than presented. This was only the starting point of an era in which several modern countries such as Portugal, Spain, Iceland, Cyprus and Greece itself declared their impossibility to repay government debt. Different studies affirm that this was a consequence of savings saturation or that it was just a condensed vision of a more magnified credit bubble \cite{lund2010debt}. Others recognize how the US housing crash triggered this crisis through different channels depending on the country of interest \cite{beker2014european}. Regardless of which is the leading cause attributable to the phenomenon and despite emphasis has always been put on the effect of the EDC on financial institutions given their obvious connection (e.g. \citeA{allegret2016impact}), contagion has characterized the stock market as a whole, and hence style factors as well.

The context being introduced, it must be noticed that this represents the downturn in which stock market was hit the least  among the selected timespans. Figure \ref{fig:figure10} adds evidence to this, with VaR and ES almost minimal among all crises, regardless of the factor of interest.

Valuation worsens with respect to the previous two disastrous years (again because financials is one of the most represented sector in the value world)\footnote{Notice also that, differently from Dataset A, the US valuation factor is very close to zero, confirming that the situation was slowly recovering in America while Europe was going through the new cycle.}, but the other sharpe ratios are all discretely positive (e.g. size and momentum). This could be the case due to two reasons. The first one is related to the fact that, as aforementioned, this represents the most not-stock-based crisis in the sample, hence helping to explain this pattern. The second reason regards the possible bias coming from the crisis-isolation issue. Being the period that was most likely to demonstrate limited catastrophic stock performances a priori, not isolating as best could implicitly consider flourishing years as well. However, there is no straightforward consensus on when this period is meant to considered started or finished.

Similarly, no significant changes are detected in terms of potential diversification perks except for the shortening of the gap between the added value brought by the style factors and the simple asset class investing.

On the other hand, the fixed income world could be of more interest, as both credit and high yield spread finally start paying off. One possible explanation boils down to the implied trade off between investing in government and corporate bonds. However, it must be reminded that, as specified in Table \ref{factors_datasets_table}, fixed income factors are based on US instruments in both Datasets, thus mining the possibility of capturing the effect of a typical European distress. Nevertheless, differently from the term spread, these two exposures are corporate bonds-based and in an environment where governments' default is behind the corner, market agents prefer to invest in firms debt instruments instead.\footnote{This is slightly paradoxical, mainly due to the fact that regardless of how much safe companies' financial stability can be perceived, the probability of government failures is lower by definition (if considering developed economies, of course).} In addition, especially in countries where structural deficit was most vulnerable, the lack of confidence led to the widening of the bond yield spreads, coherently with results not yet encountered in previously studied financial crises. 

\subsection{A close eye on the COVID-19 crisis}

The only crisis for which the minimum 2-years data rule has been avoided was the COVID-19 Pandemic, for which c.7 months of returns have been taken into consideration. Therefore, Table \ref{Covid_RR_table} below summarizes the risk-return performance of the selected factors on a semi-annualized basis (i.e. half year of market data). In fact, by simply annualizing by 252 we would implicitly assume that the effects of the pandemic are lasting the whole year and, with the benefit of hindsight, we know this could not the case.\footnote{This is also the reason why this financial distress period has been treated on a separate basis.} In addition, following the article from \citeA{Relying_factors} analyzing the behavior of factors during the first COVID-19-months only, Figure \ref{iShare_ETFs} shows the excess performance of the iShare factors EFTs over the broad American market, in our more extended timeframe. Finally, in order to have a more comprehensive overview of this turbulence, also Kenneth French's data have been analyzed, with a particular focus on momentum, valuation and size during seven critical months of the six considered turmoils (Figures \ref{MOM}, \ref{HML} and \ref{SMB}, respectively). 

To sum up these different sources of information, four are the most noticeable results:

\begin{enumerate}
\item As it is possible to appreciate from Figure \ref{iShare_ETFs} and Table \ref{Covid_RR_table}, Momentum has outperformed and it is expected to amplify this trend in future months. Not only did it survive the critical conditions of the first months, but it also seems to be recovering at a fast pace.
\item The two core factors on which literature has based its primordial discoveries, i.e. size and valuation, continue to suffer. In fact, also considering French's dataset in specific downturning months for the six crises, the period from January 2020 to June 2020 is the worst peak for value, with investors more frequently wondering what is the best way to invest in it (\citeA{Value_dormant}, Figure \ref{HML}). The intensity of size is instead intermediate if comparing with the other crises, even if the semi-annualized sharpe ratio remains strongly negative in both Dataset A and B.
\item Despite outperforming in the booming months, minimum volatility is demonstrating signs of fragility. The defensive nature of minimum volatility seems to be holding during the worst months only, when undertaking tolerable risk is crucial. On the other hand, as soon as a sort of recovery started taking place, minimum volatility has underperformed the broad US market.
\item Once again, lower correlation justifies more diversification benefits when investing in risk factors rather than in asset classes directly, coherently with past literature \cite{clarke2005factor}. Table \ref{Covid_corr_table} shows that investing in a combination of equity style factors could represent a valid strategy and that systematic risk factors are suitable alternatives for optimal portfolio optimization.
\end{enumerate}

Symmetrically to these four outcomes, four correspondent concerns must be raised:

\begin{enumerate}
\item As always, it must be remembered that momentum guarantees consistently greater returns than other styles by definition. Figure \ref{MOM} underlines how, even if levels were not as tremendous as in 2008, also the best performing stocks have been hit. The recent worldwide increase of cases and the instability associated to both the US elections and Brexit (no-Brexit), do not necessarily guarantee a bright future.
\item It is usually the case that, after the worst performing years of a specific factor, more prosperous periods follow, instance that particularly holds for valuation. Once again and despite the trend does not seem to improve at all, it would be too extreme to define this factor as completely obsolete. On the other hand, suspicions are raised regarding size. Small companies do not perform well during downturns, mostly because they are more exposed to economic cycles and less autonomous, as global financial conditions struggle to be favorable for more consolidated companies, let alone smaller realities.
\item In the first months of 2020, the shift in momentum composition from riskier to less risky stocks led to minimum volatility following the same path for that period (i.e. greater correlation between the two factors). However, this is usually a coherent behavior for a limited timespan, as momentum composition tends to go back to riskier stocks (hence higher returns) once symptoms of recovery are manifested.
\item While equity styles tend to be negatively correlated among each other, the inclusion of fixed-income styles (in particular credit and high yield spread) could determine a loss in diversification perks (see correlation between the latter and size and valuation).  
\end{enumerate}

\begin{table}[H]
\caption{Risk-return performance analysis of Datasets A and B's exposures, January 2020 - July 2020.}
\label{Covid_RR_table}
\resizebox{\textwidth}{!}{%
\begin{tabular}{@{}llccccccc@{}}
\toprule
\textit{Exposure}             & Risk premium      & \begin{tabular}[c]{@{}c@{}}Semi-Annualized\\ Return\end{tabular} & \begin{tabular}[c]{@{}c@{}}Semi-Annualized\\ Volatility\end{tabular} & \begin{tabular}[c]{@{}c@{}}Sharpe\\ Ratio\end{tabular} & \begin{tabular}[c]{@{}c@{}}Parametric\\ 95\% VaR\end{tabular} & \begin{tabular}[c]{@{}c@{}}Parametric\\ 95\% ES\end{tabular} & \begin{tabular}[c]{@{}c@{}}Historical\\ 95\% VaR\end{tabular} & \begin{tabular}[c]{@{}c@{}}Historical\\ 95\%ES\end{tabular} \\ \midrule
\textit{Asset Class (Equity)} & MSCI EAFE         & (10.10\%)                                                        & 20.81\%                                                              & (0.49)                                                 & 3.12\%                                                        & 5.48\%                                                       & 2.86\%                                                        & 4.78\%                                                      \\ \midrule
                              & MSCI JAPAN        & (8.82\%)                                                         & 18.64\%                                                              & (0.47)                                                 & 2.79\%                                                        & 4.12\%                                                       & 2.39\%                                                        & 3.91\%                                                      \\
                              & MSCI USA          & (0.12\%)                                                         & 31.08\%                                                              & 0.00                                                   & 4.52\%                                                        & 7.56\%                                                       & 4.42\%                                                        & 7.12\%                                                      \\
                              & MSCI EM           & (3.72\%)                                                         & 20.86\%                                                              & (0.18)                                                 & 3.07\%                                                        & 5.50\%                                                       & 2.67\%                                                        & 5.10\%                                                      \\ \midrule
\textit{Asset Class (F.I.)}   & Citi World Bond   & 4.60\%                                                           & 5.83\%                                                               & 0.79                                                   & 0.82\%                                                        & 1.77\%                                                       & 0.56\%                                                        & 1.45\%                                                      \\ \midrule
\textit{Style (Equity)}       & Market            & (3.56\%)                                                         & 26.30\%                                                              & (0.14)                                                 & 3.86\%                                                        & 6.77\%                                                       & 3.63\%                                                        & 6.36\%                                                      \\
\textit{}                     &                   & {\color[HTML]{303498} (1.35\%)}                                  & {\color[HTML]{303498} 31.51\%}                                       & {\color[HTML]{303498} (0.04)}                          & {\color[HTML]{303498} 4.59\%}                                 & {\color[HTML]{303498} 7.35\%}                                & {\color[HTML]{303498} 4.38\%}                                 & {\color[HTML]{303498} 7.35\%}                               \\
                              & Valuation         & (23.45\%)                                                        & 8.91\%                                                               & (2.63)                                                 & 1.51\%                                                        & 1.76\%                                                       & 1.59\%                                                        & 1.81\%                                                      \\
                              &                   & {\color[HTML]{303498} (24.86\%)}                                 & {\color[HTML]{303498} 10.56\%}                                       & {\color[HTML]{303498} (2.35)}                          & {\color[HTML]{303498} 1.77\%}                                 & {\color[HTML]{303498} 2.09\%}                                & {\color[HTML]{303498} 1.82\%}                                 & {\color[HTML]{303498} 2.18\%}                               \\
                              & Size              & (9.04\%)                                                         & 8.21\%                                                               & (1.10)                                                 & 1.28\%                                                        & 1.90\%                                                       & 1.22\%                                                        & 1.81\%                                                      \\
                              &                   & {\color[HTML]{303498} (11.73\%)}                                 & {\color[HTML]{303498} 13.69\%}                                       & {\color[HTML]{303498} (0.86)}                          & {\color[HTML]{303498} 2.10\%}                                 & {\color[HTML]{303498} 3.06\%}                                & {\color[HTML]{303498} 2.01\%}                                 & {\color[HTML]{303498} 2.64\%}                               \\
                              & Momentum          & 12.01\%                                                          & 6.35\%                                                               & 1.89                                                   & 0.84\%                                                        & 1.46\%                                                       & 0.73\%                                                        & 1.36\%                                                      \\
                              &                   & {\color[HTML]{303498} 9.59\%}                                    & {\color[HTML]{303498} 7.72\%}                                        & {\color[HTML]{303498} 1.24}                            & {\color[HTML]{303498} 1.06\%}                                 & {\color[HTML]{303498} 1.42\%}                                & {\color[HTML]{303498} 1.08\%}                                 & {\color[HTML]{303498} 1.47\%}                               \\
                              & Low Vol           & (4.37\%)                                                         & 8.09\%                                                               & (0.54)                                                 & 1.22\%                                                        & 1.95\%                                                       & 1.09\%                                                        & 1.73\%                                                      \\
                              &                   & {\color[HTML]{303498} (6.78\%)}                                  & {\color[HTML]{303498} 7.08\%}                                        & {\color[HTML]{303498} (0.96)}                          & {\color[HTML]{303498} 1.09\%}                                 & {\color[HTML]{303498} 1.67\%}                                & {\color[HTML]{303498} 0.98\%}                                 & {\color[HTML]{303498} 1.67\%}                               \\
                              & Quality           & 6.72\%                                                           & 4.56\%                                                               & 1.47                                                   & 0.62\%                                                        & 0.86\%                                                       & 0.62\%                                                        & 0.86\%                                                      \\
                              &                   & {\color[HTML]{303498} 2.89\%}                                    & {\color[HTML]{303498} 4.09\%}                                        & {\color[HTML]{303498} 0.71}                            & {\color[HTML]{303498} 0.58\%}                                 & {\color[HTML]{303498} 0.90\%}                                & {\color[HTML]{303498} 0.54\%}                                 & {\color[HTML]{303498} 0.85\%}                               \\ \midrule
\textit{Style (F.I.)}         & Credit Spread     & (1.99\%)                                                         & 2.88\%                                                               & (0.69)                                                 & 0.44\%                                                        & 0.89\%                                                       & 0.40\%                                                        & 0.82\%                                                      \\
\textit{}                     &                   & {\color[HTML]{303498} (2.33\%)}                                  & {\color[HTML]{303498} 6.87\%}                                        & {\color[HTML]{303498} (0.34)}                          & {\color[HTML]{303498} 1.02\%}                                 & {\color[HTML]{303498} 2.11\%}                                & {\color[HTML]{303498} 0.85\%}                                 & {\color[HTML]{303498} 1.94\%}                               \\
                              & High Yield Spread & (9.09\%)                                                         & 10.90\%                                                              & (0.83)                                                 & 1.67\%                                                        & 2.89\%                                                       & 1.60\%                                                        & 2.89\%                                                      \\
                              &                   & {\color[HTML]{303498} (8.01\%)}                                  & {\color[HTML]{303498} 7.58\%}                                        & {\color[HTML]{303498} (1.06)}                          & {\color[HTML]{303498} 1.17\%}                                 & {\color[HTML]{303498} 2.15\%}                                & {\color[HTML]{303498} 0.93\%}                                 & {\color[HTML]{303498} 2.00\%}                               \\
                              & Term Spread       & 17.92\%                                                          & 17.15\%                                                              & 1.04                                                   & 2.37\%                                                        & 4.96\%                                                       & 1.52\%                                                        & 3.61\%                                                      \\
                              &                   & {\color[HTML]{303498} 18.27\%}                                   & {\color[HTML]{303498} 17.35\%}                                       & {\color[HTML]{303498} 1.05}                            & {\color[HTML]{303498} 2.40\%}                                 & {\color[HTML]{303498} 4.37\%}                                & {\color[HTML]{303498} 1.57\%}                                 & {\color[HTML]{303498} 3.58\%}                               \\ \bottomrule
\end{tabular}}
\medskip
\caption*{\textit{Notes}: For style factors: Dataset A values in black, Dataset B values in blue.}
\end{table}

\text{\\}

\begin{figure}[H]
\begin{minipage}[b]{.5\textwidth}
\centering
\caption{iShares Edge MSCI USA Size, Value, Min Vol, Momentum, Quality ETFs' excess returns vs. S\&P500, from January 1, 2020 to July 10, 2020.}
\includegraphics[scale=0.55]{ETFs.pdf}
\label{iShare_ETFs}
\end{minipage}
\hfill
\begin{minipage}[b]{.5\textwidth}
\centering
\captionof{figure}{Daily MOM returns normalized to 100 for crucial months of the selected crises.}
\includegraphics[scale=0.55]{FF_MOMENTUM.pdf}
\label{MOM}
\end{minipage}
\end{figure}

\begin{figure}[H]
\begin{minipage}[b]{.5\textwidth}
\centering
\captionof{figure}{Daily HML returns normalized to 100 for crucial months of the selected crises.}
\includegraphics[scale=0.55]{FF_VALUATION.pdf}
\label{HML}
\end{minipage}%
\hfill
\begin{minipage}[b]{.5\textwidth}
\centering
\captionof{figure}{Daily SMB returns normalized to 100 for crucial months of the selected crises.}
\includegraphics[scale=0.55]{FF_SIZE.pdf}
\label{SMB}
\end{minipage}
\caption*{\textit{Source}: \url{http://mba.tuck.dartmouth.edu/pages/faculty/ken.french/data_library.html}, Bloomberg as July 2020 (iShare).}
\end{figure}

\text{\\}

\begin{table}[H]
\begin{center}
\caption{Correlations of Datasets A and B's style and asset class premia, January 2020 - July 2020.}
\label{Covid_corr_table}
\resizebox{0.65\textwidth}{!}{%
\begin{tabular}{@{}lccccccccc@{}}
\toprule
\multicolumn{1}{c}{}  & Market                                 & Valuation                              & Size                                   & Momentum                               & Low Volatility                         & Quality                                & Credit Spread                          & High Yield Spread                      & Term Spread                            \\ \midrule
Market                & 1.00                                   & \textbf{}                              & \textbf{}                              & \textbf{}                              & \textbf{}                              & \textbf{}                              & \textbf{}                              & \textbf{}                              & \textbf{}                              \\
                      & {\color[HTML]{303498} 1.00}            & \multicolumn{1}{l}{}                   & \multicolumn{1}{l}{}                   & \multicolumn{1}{l}{}                   & \multicolumn{1}{l}{}                   & \multicolumn{1}{l}{}                   & \multicolumn{1}{l}{}                   & \multicolumn{1}{l}{}                   & \multicolumn{1}{l}{}                   \\
Valuation             & \textbf{0.31}                          & 1.00                                   & \textbf{}                              & \textbf{}                              & \textbf{}                              & \textbf{}                              & \textbf{}                              & \textbf{}                              & \textbf{}                              \\
                      & {\color[HTML]{303498} \textbf{0.22}}   & {\color[HTML]{303498} 1.00}            & \multicolumn{1}{l}{}                   & \multicolumn{1}{l}{}                   & \multicolumn{1}{l}{}                   & \multicolumn{1}{l}{}                   & \multicolumn{1}{l}{}                   & \multicolumn{1}{l}{}                   & \multicolumn{1}{l}{}                   \\
Size                  & \textbf{0.15}                          & \textbf{0.49}                          & 1.00                                   & \textbf{}                              & \textbf{}                              & \textbf{}                              & \textbf{}                              & \textbf{}                              & \textbf{}                              \\
                      & {\color[HTML]{303498} \textbf{0.35}}   & {\color[HTML]{303498} \textbf{0.55}}   & {\color[HTML]{303498} 1.00}            & \multicolumn{1}{l}{}                   & \multicolumn{1}{l}{}                   & \multicolumn{1}{l}{}                   & \multicolumn{1}{l}{}                   & \multicolumn{1}{l}{}                   & \multicolumn{1}{l}{}                   \\
Momentum              & \textbf{(0.25)}                        & \textbf{(0.71)}                        & \textbf{(0.45)}                        & 1.00                                   & \textbf{}                              & \textbf{}                              & \textbf{}                              & \textbf{}                              & \textbf{}                              \\
                      & {\color[HTML]{303498} \textbf{(0.16)}} & {\color[HTML]{303498} \textbf{(0.52)}} & {\color[HTML]{303498} \textbf{(0.48)}} & {\color[HTML]{303498} 1.00}            & \multicolumn{1}{l}{}                   & \multicolumn{1}{l}{}                   & \multicolumn{1}{l}{}                   & \multicolumn{1}{l}{}                   & \multicolumn{1}{l}{}                   \\
Low Volatility        & \textbf{(0.69)}                        & \textbf{(0.13)}                        & \textbf{(0.16)}                        & 0.47                                   & 1.00                                   & \textbf{}                              & \textbf{}                              & \textbf{}                              & \textbf{}                              \\
                      & {\color[HTML]{303498} \textbf{(0.42)}} & {\color[HTML]{303498} \textbf{0.24}}   & {\color[HTML]{303498} \textbf{(0.19)}} & {\color[HTML]{303498} 0.38}            & {\color[HTML]{303498} 1.00}            & \multicolumn{1}{l}{}                   & \multicolumn{1}{l}{}                   & \multicolumn{1}{l}{}                   & \multicolumn{1}{l}{}                   \\
Quality               & \textbf{(0.06)}                        & \textbf{(0.64)}                        & \textbf{(0.77)}                        & 0.50                                   & \textbf{0.02}                          & 1.00                                   & \textbf{}                              & \textbf{}                              & \textbf{}                              \\
                      & {\color[HTML]{303498} \textbf{(0.17)}} & {\color[HTML]{303498} \textbf{(0.57)}} & {\color[HTML]{303498} \textbf{(0.63)}} & {\color[HTML]{303498} 0.41}            & {\color[HTML]{303498} \textbf{0.03}}   & {\color[HTML]{303498} 1.00}            & \multicolumn{1}{l}{}                   & \multicolumn{1}{l}{}                   & \multicolumn{1}{l}{}                   \\ \midrule
Credit Spread         & 0.61                                   & \textbf{0.38}                          & \textbf{0.13}                          & \textbf{(0.29)}                        & \textbf{(0.40)}                        & \textbf{(0.20)}                        & 1.00                                   & \textbf{}                              & \textbf{}                              \\
                      & {\color[HTML]{303498} 0.55}            & {\color[HTML]{303498} \textbf{0.32}}   & {\color[HTML]{303498} \textbf{0.16}}   & {\color[HTML]{303498} \textbf{(0.00)}} & {\color[HTML]{303498} \textbf{(0.00)}} & {\color[HTML]{303498} \textbf{(0.28)}} & {\color[HTML]{303498} 1.00}            & \multicolumn{1}{l}{}                   & \multicolumn{1}{l}{}                   \\
High Yield Spread     & 0.76                                   & \textbf{0.36}                          & \textbf{0.21}                          & \textbf{(0.28)}                        & \textbf{(0.48)}                        & \textbf{(0.20)}                        & 0.86                                   & 1.00                                   & \textbf{}                              \\
                      & {\color[HTML]{303498} 0.58}            & {\color[HTML]{303498} \textbf{0.23}}   & {\color[HTML]{303498} \textbf{0.12}}   & {\color[HTML]{303498} \textbf{(0.18)}} & {\color[HTML]{303498} \textbf{(0.24)}} & {\color[HTML]{303498} \textbf{(0.23)}} & {\color[HTML]{303498} \textbf{0.44}}   & {\color[HTML]{303498} 1.00}            & \multicolumn{1}{l}{}                   \\
Term Spread           & \textbf{(0.40)}                        & \textbf{(0.21)}                        & \textbf{0.16}                          & \textbf{0.26}                          & 0.35                                   & \textbf{(0.09)}                        & \textbf{(0.66)}                        & \textbf{(0.62)}                        & 1.00                                   \\
                      & {\color[HTML]{303498} \textbf{(0.43)}} & {\color[HTML]{303498} \textbf{(0.14)}} & {\color[HTML]{303498} \textbf{0.02}}   & {\color[HTML]{303498} \textbf{0.15}}   & {\color[HTML]{303498} 0.26}            & {\color[HTML]{303498} \textbf{(0.06)}} & {\color[HTML]{303498} \textbf{(0.26)}} & {\color[HTML]{303498} \textbf{(0.82)}} & {\color[HTML]{303498} 1.00}            \\ \midrule
MSCI EAFE             & 0.82                                   & \textbf{0.42}                          & \textbf{0.39}                          & \textbf{(0.30)}                        & \textbf{(0.53)}                        & \textbf{(0.41)}                        & 0.62                                   & 0.71                                   & \textbf{(0.26)}                        \\
                      & {\color[HTML]{303498} 0.71}            & {\color[HTML]{303498} \textbf{0.36}}   & {\color[HTML]{303498} \textbf{0.38}}   & {\color[HTML]{303498} \textbf{(0.16)}} & {\color[HTML]{303498} \textbf{(0.18)}} & {\color[HTML]{303498} \textbf{(0.35)}} & {\color[HTML]{303498} 0.67}            & {\color[HTML]{303498} 0.53}            & {\color[HTML]{303498} \textbf{(0.29)}} \\
MSCI JAPAN            & 0.46                                   & \textbf{0.26}                          & \textbf{0.32}                          & \textbf{(0.21)}                        & \textbf{(0.24)}                        & \textbf{(0.44)}                        & 0.48                                   & 0.45                                   & \textbf{(0.28)}                        \\
                      & {\color[HTML]{303498} \textbf{0.36}}   & {\color[HTML]{303498} \textbf{0.19}}   & {\color[HTML]{303498} \textbf{0.18}}   & {\color[HTML]{303498} \textbf{0.00}}   & {\color[HTML]{303498} \textbf{(0.05)}} & {\color[HTML]{303498} \textbf{(0.26)}} & {\color[HTML]{303498} 0.44}            & {\color[HTML]{303498} 0.40}            & {\color[HTML]{303498} \textbf{(0.30)}} \\
MSCI USA              & 0.98                                   & \textbf{0.24}                          & \textbf{0.05}                          & \textbf{(0.20)}                        & \textbf{(0.68)}                        & \textbf{0.08}                          & 0.56                                   & 0.72                                   & \textbf{(0.42)}                        \\
                      & {\color[HTML]{303498} 1.00}            & {\color[HTML]{303498} \textbf{0.20}}   & {\color[HTML]{303498} \textbf{0.31}}   & {\color[HTML]{303498} \textbf{(0.14)}} & {\color[HTML]{303498} \textbf{(0.41)}} & {\color[HTML]{303498} \textbf{(0.14)}} & {\color[HTML]{303498} 0.55}            & {\color[HTML]{303498} 0.58}            & {\color[HTML]{303498} \textbf{(0.44)}} \\
MSCI EM               & 0.72                                   & \textbf{0.32}                          & \textbf{0.27}                          & \textbf{(0.28)}                        & \textbf{(0.56)}                        & \textbf{(0.33)}                        & 0.69                                   & 0.72                                   & \textbf{(0.32)}                        \\
                      & {\color[HTML]{303498} 0.62}            & {\color[HTML]{303498} \textbf{0.25}}   & {\color[HTML]{303498} \textbf{0.20}}   & {\color[HTML]{303498} \textbf{(0.10)}} & {\color[HTML]{303498} \textbf{(0.26)}} & {\color[HTML]{303498} \textbf{(0.31)}} & {\color[HTML]{303498} 0.63}            & {\color[HTML]{303498} 0.60}            & {\color[HTML]{303498} \textbf{(0.34)}} \\ \midrule
Citi World Bond Index & \textbf{(0.11)}                        & \textbf{(0.01)}                        & \textbf{0.34}                          & \textbf{0.17}                          & \textbf{0.24}                          & \textbf{(0.24)}                        & \textbf{(0.29)}                        & \textbf{(0.20)}                        & 0.75                                   \\
                      & {\color[HTML]{303498} \textbf{(0.17)}} & {\color[HTML]{303498} \textbf{0.05}}   & {\color[HTML]{303498} \textbf{0.12}}   & {\color[HTML]{303498} \textbf{0.13}}   & {\color[HTML]{303498} \textbf{0.32}}   & {\color[HTML]{303498} \textbf{(0.24)}} & {\color[HTML]{303498} \textbf{0.11}}   & {\color[HTML]{303498} \textbf{(0.44)}} & {\color[HTML]{303498} 0.74}            \\ \bottomrule
\end{tabular}}
\bigskip
\caption*{\textit{Notes}: Bold denotes values lower than or equal to 0.20. Dataset A values in black, Dataset B values in blue.}
\end{center}
\end{table}

\subsection{Crisis regime analysis exercise}

First of all, the F-test of overall significance showed positive results, with p-values always near to zero in every regression. However, it must be remembered that this test examines the significance of the coefficients' estimates jointly, leaving room for conflicting outcomes. Hence, Table \ref{Significance_table} delineates a deeper overview of the significance of the analysis, as it shows the \% of significant coefficients by crisis (i.e. the \% of p-values less or equal than the maximum commonly accepted significance level) and the \% of R-squared greater or equal than a threshold of 0.15. In general, both types of indicators confirm that the analysis gains strength over time due to the improved quality of data. The right hand side of this table is crucial, as considerations will be based on the significant portions of the coefficients only. \\

\begin{table}[H]
\caption{Summary table of the analysis' significance.}
\label{Significance_table}
\resizebox{\textwidth}{!}{%
\begin{tabular}{@{}lcccccccc@{}}
\toprule
\textbf{Crisis}                & \textbf{\begin{tabular}[c]{@{}c@{}}Avg. \# of \\ observations\end{tabular}} & \textbf{\begin{tabular}[c]{@{}c@{}}\% of R-squared\\ \textgreater 0.15\end{tabular}} & \textbf{\begin{tabular}[c]{@{}c@{}}\% of significant\\ Market betas\end{tabular}} & \textbf{\begin{tabular}[c]{@{}c@{}}\% of significant\\ Valuation betas\end{tabular}} & \textbf{\begin{tabular}[c]{@{}c@{}}\% of significant\\ Size betas\end{tabular}} & \textbf{\begin{tabular}[c]{@{}c@{}}\% of significant\\ Momentum betas\end{tabular}} & \textbf{\begin{tabular}[c]{@{}c@{}}\% of significant\\ Low Volatility betas\end{tabular}} & \textbf{\begin{tabular}[c]{@{}c@{}}\% of significant\\ Quality betas\end{tabular}} \\ \midrule
Asian Crisis 1997              & 489                                                                         & 44\%                                                                                 & 94\%                                                                              & 12\%                                                                                 & 38\%                                                                            & 30\%                                                                                & 84\%                                                                                      & 62\%                                                                               \\
August 1998 turmoil            & 495                                                                         & 42\%                                                                                 & 86\%                                                                              & 10\%                                                                                 & 46\%                                                                            & 46\%                                                                                & 52\%                                                                                      & 76\%                                                                               \\
Dot-com Bubble \& 11 Sept 2001 & 722                                                                         & 44\%                                                                                 & 96\%                                                                              & 58\%                                                                                 & 60\%                                                                            & 66\%                                                                                & 72\%                                                                                      & 84\%                                                                               \\
2008 Financial Crisis          & 741                                                                         & 82\%                                                                                 & 100\%                                                                             & 70\%                                                                                 & 64\%                                                                            & 82\%                                                                                & 62\%                                                                                      & 80\%                                                                               \\
(Peak of) European Debt Crisis & 744                                                                         & 100\%                                                                                & 100\%                                                                             & 84\%                                                                                 & 78\%                                                                            & 78\%                                                                                & 78\%                                                                                      & 78\%                                                                               \\
COVID-19 Pandemic              & 747                                                                         & 84\%                                                                                 & 100\%                                                                             & 70\%                                                                                 & 56\%                                                                            & 68\%                                                                                & 82\%                                                                                      & 70\%                                                                               \\ \bottomrule
\end{tabular}}
\end{table}

Reporting each and every single coefficient would result in a too dispersive work\footnote{Considering 6 crises with 6 styles and 50 regressions per crisis would mean including a table with almost 2,100 entries.}, hence the average of the coefficients that are at least  90\% statistically significant is reported in Table \ref{avg_coefficients} below. \\

\begin{table}[H]
\caption{Average of the (at least) 90\% statistically significant coefficients.}
\label{avg_coefficients}
\resizebox{\textwidth}{!}{%
\begin{tabular}{@{}lcccccc@{}}
\toprule
\textbf{Crisis}                & \textbf{Market} & \textbf{Valuation} & \textbf{Size} & \textbf{Momentum} & \textbf{Low Volatility} & \textbf{Quality} \\ \midrule
Asian Crisis 1997              & 2.09            & 0.89               & 1.24          & (0.64)            & 2.40                    & (0.34)           \\
August 1998 turmoil            & 1.41            & 1.26               & 1.17          & 0.56              & 0.97                    & 0.33             \\
Dot-com Bubble \& 11 Sept 2001 & 1.21            & 0.46               & 0.59          & 1.08              & 0.97                    & (0.48)           \\
2008 Financial Crisis          & 1.07            & (0.04)             & 0.58          & (0.01)            & (0.26)                  & (0.75)           \\
(Peak of) European Debt Crisis & 0.79            & (1.04)             & (0.60)        & (0.60)            & (0.87)                  & (1.73)           \\
COVID-19 Pandemic              & 0.94            & 0.27               & 0.44          & (0.35)            & (0.72)                  & (0.36)           \\ \bottomrule
\end{tabular}}
\end{table}

However, despite being a synthetic metric that can help to summarize results, this does not represent an extremely accurate proxy, mainly because of the different exposure directions of the different stocks. In fact, to be more precise, what we are interested in is:
\begin{itemize}
\item Among the statistically significant coefficients, (a) how many of them were actually positive (i.e. exposed to the long leg of the factor) and (b) how many were negative (i.e. exposed to the short leg of the factor), for each factor, for each crisis.
\item Respectively, the average value of the coefficients included in the subsets (a) and (b), for each factor, for each crisis.
\end{itemize}

Table \ref{beta_exposures} below summarizes the aforementioned measures, to give a general idea on both the direction and the entity of the factor exposures.

% Please add the following required packages to your document preamble:
% \usepackage{booktabs}
\begin{table}[]
\caption{Summary of the beta exposures by crisis (only at-least-90\% statistically significant coefficients are considered).}
\label{beta_exposures}
\resizebox{\textwidth}{!}{%
\begin{tabular}{@{}clcccccc@{}}
\toprule
                               &                          & \textbf{Market} & \textbf{Valuation} & \textbf{Size} & \textbf{Momentum} & \textbf{Low Volatility} & \textbf{Quality} \\ \midrule
                               & \# Beta \textgreater 0   & 47              & 4                  & 17            & 2                 & 41                      & 9                \\
                               & \# Beta \textless 0      & 0               & 2                  & 2             & 13                & 1                       & 22               \\
Asian Crisis 1997              & Tot                      & 47              & 6                  & 19            & 15                & 42                      & 31               \\
                               & Avg. Beta \textgreater 0 & 2.09            & 2.76               & 1.47          & 1.09              & 2.50                    & 0.99             \\
                               & Avg. Beta \textless 0    & n.a.            & -2.84              & -0.74         & -0.90             & -1.81                   & -0.89            \\ \midrule
                               & \# Beta \textgreater 0   & 43              & 4                  & 21            & 17                & 23                      & 28               \\
                               & \# Beta \textless 0      & 0               & 1                  & 2             & 6                 & 3                       & 10               \\
August 1998 turmoil            & Tot                      & 43              & 5                  & 23            & 23                & 26                      & 38               \\
                               & Avg. Beta \textgreater 0 & 1.41            & 2.10               & 1.34          & 0.99              & 1.24                    & 0.72             \\
                               & Avg. Beta \textless 0    & n.a.            & -2.09              & -0.62         & -0.66             & -1.10                   & -0.73            \\ \midrule
                               & \# Beta \textgreater 0   & 48              & 23                 & 21            & 30                & 35                      & 12               \\
                               & \# Beta \textless 0      & 0               & 6                  & 9             & 3                 & 1                       & 30               \\
Dot-com Bubble \& 11 Sept 2001 & Tot                      & 48              & 29                 & 30            & 33                & 36                      & 42               \\
                               & Avg. Beta \textgreater 0 & 1.21            & 0.88               & 1.08          & 1.23              & 1.03                    & 0.67             \\
                               & Avg. Beta \textless 0    & n.a.            & -1.15              & -0.58         & -0.41             & -1.30                   & -0.94            \\ \midrule
                               & \# Beta \textgreater 0   & 50              & 19                 & 20            & 16                & 16                      & 17               \\
                               & \# Beta \textless 0      & 0               & 16                 & 12            & 25                & 15                      & 23               \\
2008 Financial Crisis          & Tot                      & 50              & 35                 & 32            & 41                & 31                      & 40               \\
                               & Avg. Beta \textgreater 0 & 1.07            & 0.96               & 1.31          & 0.79              & 1.07                    & 0.98             \\
                               & Avg. Beta \textless 0    & n.a.            & -1.23              & -0.64         & -0.52             & -1.68                   & -2.03            \\ \midrule
                               & \# Beta \textgreater 0   & 50              & 9                  & 5             & 3                 & 11                      & 6                \\
                               & \# Beta \textless 0      & 0               & 33                 & 34            & 36                & 28                      & 33               \\
(Peak of) European Debt Crisis & Tot                      & 50              & 42                 & 39            & 39                & 39                      & 39               \\
                               & Avg. Beta \textgreater 0 & 0.79            & 1.09               & 1.67          & 0.37              & 1.68                    & 0.93             \\
                               & Avg. Beta \textless 0    & n.a.            & -1.63              & -0.93         & -0.69             & -1.87                   & -2.22            \\ \midrule
                               & \# Beta \textgreater 0   & 50              & 21                 & 22            & 9                 & 10                      & 14               \\
                               & \# Beta \textless 0      & 0               & 14                 & 6             & 25                & 31                      & 21               \\
COVID-19 Pandemic              & Tot                      & 50              & 35                 & 28            & 34                & 41                      & 35               \\
\multicolumn{1}{l}{}           & Avg. Beta \textgreater 0 & 0.94            & 1.01               & 0.68          & 0.93              & 0.71                    & 0.95             \\
\multicolumn{1}{l}{}           & Avg. Beta \textless 0    & n.a.            & -0.84              & -0.45         & -0.81             & -1.18                   & -1.23            \\ \bottomrule
\end{tabular}}
\end{table}

Oppositely from the previous section, it is not optimal to recapitulate the findings of this analysis by crisis. In fact, as underlined before, we must be particularly careful when discussing styles in lights of the Asian turbulence of 1997, as the static approach is being utilized in contrast to the other debacles.

A further important information that is critical in order to discuss factor exposures (regardless of which style is being analyzed) is the geographical and sectorial split of the involved firms, which is possible to appreciate in Figure \ref{worst_geo} and \ref{worst_sector} below. In general, results are in line with the nature of the crises from both standpoints: in 1997, significant was the number of Asia Pacific companies representing the worst performing stocks during the faced tumults\footnote{In particular, out of those 29: 21 Japan, 5 Hong Kong, 2 Singapore, 1 Australia.}; 28 companies out of 50 were European among them in 2011, with the US suffering less from European government's defaults and keeping their post-Lehmann recovery in place; properly discussing Lehmann, half of the worst performing companies were in the financial industry in 2008, and so on. These outcomes will help explaining both the patterns cross-crisis and finding coherence / incoherence between this and the previous section. \\ \\

\begin{figure}[H]
\begin{center}
\caption{Geographical split of the iShare MSCI World ETF's worst performing stocks, by crisis.}
\includegraphics[scale=0.7]{WORST_STOCKS_GEO.pdf}
\caption*{\textit{Source}: Bloomberg and analysis of the author.}
\label{worst_geo}
\end{center}
\end{figure}

\begin{figure}[H]
\begin{center}
\caption{Sectorial split of the iShare MSCI World ETF's worst performing stocks, by crisis.}
\includegraphics[scale=0.7]{WORST_STOCKS_SECTORS.pdf}
\caption*{\textit{Source}: Bloomberg and analysis of the author.}
\label{worst_sector}
\end{center}
\end{figure}

Considerations for the market as a whole are frivolous, as this represents the easiest factor to be studied and the only one for which values in Table \ref{avg_coefficients} can be already considered as fairly acceptable metrics. In fact, even if computed as the difference between a global index and a proxy for the risk free rate, we cannot define the market as a pure long-short portfolio as done for the other factors. Signs are always positive (coinciding values in both Table \ref{avg_coefficients} and \ref{beta_exposures} ) and coefficients are almost always significant regardless of the crisis, symptom that excess returns will always be partly explained by the exposure to the stock market itself. Despite being trivial, this result plays the crucial role of being a starting measure on which next discussions are borne.\footnote{Not having these conclusions would have meant going against the milestone pieces of literature on which modern Asset Pricing Theory itself is based.}

The empirical exercise does not allow us to conclude anything in particular for the exposure to valuation in the first two timespans. As already anticipated in Table \ref{Significance_table}, the number of significant coefficients is not enough to draw conclusions.

However, the context is extremely interesting in the other four,  as the European Debt Crisis sees 33 out 42 of the selected stocks to be negatively exposed to valuation, whereas this number boils down to only 6 out of 29 in the Dot-Com Bubble. On the other hand, the situation is more stable and balanced in 2008 and 2020, with a slight win for the positive betas in both. Hence, it is true that companies turning out to be the worst performing stocks in the analyzed distress periods were heterogeneously exposed to either value or growth. However, with the exception of the European Debt Crisis, results are coherent with what studied in the risk-return performance section, as these stocks did not excel and they were positively exposed to valuation. 

Following the scrutiny done in \textit{Subsection \ref{whole_period} - \nameref{whole_period}}, it is possible to notice that the presence of energy and financials was significant in the most recent turbulences. As anticipated, these are often the industries in which we find value stocks, justifying the positive exposure to this factor in 2001, 2008 and 2020. For the same reasons, a major component of tech stocks would have been expected to see their share price collapsing in 2011, where worst performing stocks were more exposed to growth rather than value. However, this does not seem to be the case, as companies were rather skewed to utilities (mostly electricity companies such as Electricite de France or Tokyo Electric Power).

Overall, value firms represent a non negligible portion of the worst performing stocks in the recent analyzed periods, with the exception of the European Debt Crisis, where these would have guaranteed higher returns rather than growth stocks.

The situation is different when pondering size, as numbers are sustainable enough to start a discussion also for the oldest two crises. Moreover, Table \ref{avg_mkt_cap} describes the average market cap of the selected companies at the starting and ending date of the selected pre-crisis period,  in USDm (considering the market cap. at the starting date only would have biased the analysis, because it would have not considered possible capital increases happening during the timespan).\footnote{A bias would have been present also if considering the current market capitalization, for the different reason that it would have not been coherent with the selected period.} If defining small and mid cap companies to be firms with market capitalization less than USD2bn and between USD2bn and USD10bn respectively, we notice how both Tables \ref{beta_exposures} and \ref{avg_mkt_cap} indicate that these companies were mostly exposed to the size factor, slightly going against what empirically found in the past. In fact, it must be stressed how these considerations are based on the fact that the sample is composed of poor performing stocks. By saying that these are positively exposed to size, where size is defined as a portfolio long small/mid cap companies and short big ones, we are actually saying that small/mid cap companies are guaranteeing the worst returns in that specific crisis, which is not what empirical evidence has shown over time. The opposite judgement holds when discussing the European Debt crisis, as the numbers of significant  size betas are negative, implying that larger companies were suffering the most in that timespan. \\

\begin{table}[H]
\centering
\caption{Average market cap. of the worst performing stocks of the selected crises. Values in USDm.}
\label{avg_mkt_cap}
\resizebox{0.7\textwidth}{!}{%
\begin{tabular}{@{}lccc@{}}
\toprule
                & \begin{tabular}[c]{@{}c@{}}Avg. Market Cap\\ (Starting date)\end{tabular} & \begin{tabular}[c]{@{}c@{}}Avg. Market Cap\\ (Ending date)\end{tabular} & Average \\ \midrule
Asian Crisis 1997              & 5,296                                                                     & 2,675                                                                   & 3,986   \\
August 1998 turmoil            & 3,783                                                                     & 5,750                                                                   & 4,766   \\
Dot-com Bubble \& 11 Sept 2001 & 16,089                                                                    & 29,636                                                                  & 22,862  \\
2008 Financial Crisis          & 34,866                                                                    & 49,343                                                                  & 42,104  \\
(Peak of) European Debt Crisis & 40,144                                                                    & 36,120                                                                  & 38,132  \\
COVID-19 Pandemic              & 34,395                                                                    & 34,423                                                                  & 34,409  \\ \bottomrule
\end{tabular}}
\end{table}

In this, the nature of the crisis keeps playing a crucial role. For instance, with all the changes generated in employee's daily life, COVID-19 has put a strain on the ability to continue normal business activities for small companies, whereas larger companies are usually more prepared to sustain these kinds of events. Catastrophes like these are classified as the so-called LFHI events, i.e. events with low frequency (low probability), but generating high impact (severity) \cite{sironi2007risk}. Larger companies are usually more structured and organized to sustain these risks, with a quicker reaction time and a greater degree of adaptability to significant changes (changes that could go from remote working for COVID19, to the annulment of an already defined target acquisition, possibly due to banks failure to provide funds in the 2008 Financial Crisis). Smaller companies struggle and require more time to mitigate the changes, justifying these results.

Momentum's exposure confirms the theory behind it for most crises, i.e. the fact that worst performing stocks are negatively exposed to a style that instead tends to mimic the best ones. However, both the 1998 Turmoil and the Dot-com Bubble behave in the opposite way, with more positive coefficients than negative ones. Similar considerations hold for low volatility and quality, especially for the latter: in five cases out of six the exposure is negative, meaning that it was indeed junk companies to become the worst performing stocks in those specific turbulences.

Of course, one last but crucial consideration regards the correlations among the different independent variables. As anticipated in the discussion of the risk-return performance of the Dot-com Bubble, having negative correlation does not necessarily help in this context (so does hold for positive correlation). In fact, here we are trying to explain what drives the excess returns of the firms in that specific historical period, whereas in the previous section the idea was to underline the possible benefits coming from investing in different factors. Regarding the crisis regime analysis, results for the six correlations tables for the six sets of crisis data are ambiguous. Market and valuation's correlation with the other factors are not generally too far away from zero. However, momentum displays high correlation with quality and low volatility. These two are also very correlated, but this does not surprise as low volatility has historically been associated to a proxy for quality itself.

It must be said that, also in this area of analysis, the quality of data remains fundamental: despite not being completely null, correlation approaches zero from both sides in more recent crises, while numbers are greater if going back in the past.

% Conclusion paragraph:
\newpage
\section{Conclusions}

\subsection{Style factors during different types of crises}

Merging the results obtained both in the risk-return performance analysis and in the empirical exercise, four are the most intriguing findings of this work:

\begin{enumerate}
\item \textit{Heterogeneity of style factors on a cross-crisis basis, but...}. Both the risk-return performance and the crisis regime analysis methodology demonstrate how, despite they all affect financial markets in a terrible way, the nature of the crisis plays a crucial role also when investing through equity styles. Being a pure financial crisis and being a financial crisis induced by the consequences of the spread of a lethal virus are two different instances, especially when it comes to understanding excess returns' drivers.

\item \textit{... some patterns can still be identified.} Is value investing dead? No, but signals are not of the brightest ones as value stocks keep suffering more than others during the studied financial crises, regardless of the nature of the latter. Will small cap companies continue to generate excess returns in the future? Probably, but whenever a dropdown is behind the corner investors should keep caution, maybe considering directing their funds to more affirmed companies, usually more able to sustain the consequences easier and quicker.

\item \textit{Factor-based investing vs. asset class investing.} This work confirms what previously anticipated by past literature and extends it to financial drop-downs, i.e. that styles can be considered as valid alternative investment strategies, able to generate excess returns and especially to exploit diversification benefits that would not be exploited otherwise. In particular, this also holds for the darkest financial periods of the last thirty years, but to appreciate these perks investors should acknowledge the possibility of investing through these risk factors, rather than focusing only on the simple equity premium. Indeed, this should be done by analyzing these alternative factors at the same time, because switching investments from one to another would mine the whole concept of factor investing itself \cite{asness2015investing}. 
Does this mean that the classic asset class investment approach is to be believed obsolete? The answer remains no, for multiple reasons related to the aforementioned \textit{three dirty words in finance}: leverage, short-selling and the usage of derivatives cannot be left aside in a context in which building long-short portfolios is the core practice.

\item \textit{COVID-19.} Factor investing has been hit by the COVID-19 pandemic offset as this was inevitable, but numbers seem to be even worse than 2008. In the first months, defensive factors have helped to mitigate consequences, but April-July remains a period in which skepticism has driven market pricing. Valuation is not the only impacted factor. Will a new lockdown worsen styles performance? Will momentum and minimum volatility's correlation determine a safe harbor for investors? Unfortunately, the answer is certain and positive only to the first question, because both micro and macro future events will indirectly shape styles' performance.

\end{enumerate}

\subsection{Limits and follow-up research} \label{limitations}
Four were also the main areas of improvement identified during and after the analysis, all of them opening to further research and possible development of the work itself:

\begin{enumerate}
\item First of all, \textit{the construction process of the factors}. It is true that a fraction of past literature implements the long-short portfolio construction using already-built indices (i.e. as done in this work). However, the best way to define them would be to singularly select the stocks based on the core characteristics of the factors themselves. For example, the absence of a "High Volatility Index" prevents to create a factor which is technically long the low beta exposure and short the opposite one. \citeA{frazzini2014betting} construct this factor by ranking companies with low or high beta. \citeA{fama1992cross} did this for the first time with value and size, by building the SMB (small-minus-big) and HML (high-minus-low) portfolios on which their database is based. One further research could be to re-run the analysis and the empirical exercise by building each and every one of the factors in this way, so to start even one step behind and see if significant differences are verified. 
\item Secondly, \textit{the benefit of hindsight}. In order to process the analysis and exercises done in this thesis, an ending data had to be selected (i.e. July 10, 2020), date after which no further data would be collected anymore. However, time passed, milestones have been achieved and the global situation has changed. Running the analysis now that the COVID-19 crisis has evolved could possibly shed lights on some of the arisen doubts.
\item \textit{Low quality data for not recent prices}. The less contemporary the financial distress period, the greater the bias coming from data retrievability. Even if in different manners, this aspect was suffered both in the risk-return performance analysis and in the event methodology, notching the quality and reliability of the Asian and 1998's results. A further improvement could be to amend the case study by considering a lagged window also for the Asian crisis, but excluding style factors for which data are not available. However, due to its nature, this last issue does not have a direct solution, despite building factors as described in the first point could help to overcome it.
\item Finally, \textit{the inclusion of an ESG factor} could be a point of further discussion. Nowadays, companies are more and more aware of sustainability issues and it could be interesting to study how ESG-based factor investing behaved during the analyzed downturns.\footnote{Of course, this should be done by putting more emphasis on the patterns during the most recent crises, given an anticipated issue of data retrievability due to the fact that ESG themes have become substantial only in the last decade.}
\end{enumerate}

% Appendix:
\newpage
\appendix
\appendixpage
\addappheadtotoc

\begin{figure}[h]
\begin{center}
\caption{Value vs growth, 1997-2020.}
\label{VALUEGROWTH}
\includegraphics[scale=0.6]{VALUEGROWTH.pdf}
\caption*{\textit{Source}: Bloomberg as of July 10, 2020.}
\end{center}
\end{figure}

\begin{table}[H]
\begin{center}
\caption{Equity style factors' variance inflation factor (VIF) for the Dot-com bubble.}
\label{VIF_dot-com}
\resizebox{0.2\textwidth}{!}{%
\begin{tabular}{@{}lll@{}}
\toprule
Variable      & VIF  & 1/VIF \\ \midrule
LowVolatility & 3.80 & 0.26  \\
Market        & 3.50 & 0.29  \\
Valuation     & 2.42 & 0.41  \\
Size          & 1.77 & 0.57  \\
Momentum      & 1.56 & 0.64  \\
Quality       & 1.55 & 0.64  \\ \midrule
Mean VIF      & 2.43 &       \\ \bottomrule
\end{tabular}}
\end{center}
\end{table}

\begin{table}[H]
\begin{center}
\caption{Betas, adjusted betas and average market cap of the selected Dot-com bubble notable companies.}
\label{Betas_market_caps_dot_com}
\resizebox{\textwidth}{!}{%
\begin{tabular}{@{}lccccccccccc@{}}
\toprule
                     & \multicolumn{1}{l}{3Com} & \multicolumn{1}{l}{Akamai Technologies} & \multicolumn{1}{l}{Blue Coat Systems} & \multicolumn{1}{l}{Blucora} & \multicolumn{1}{l}{Digex} & \multicolumn{1}{l}{SAVVIS} & \multicolumn{1}{l}{Terra Networks} & \multicolumn{1}{l}{Inktomi} & \multicolumn{1}{l}{TIBCO Software} & \multicolumn{1}{l}{Covad} & \multicolumn{1}{l}{Average} \\ \midrule
Beta                 & 1.47***                  & 3.25***                                 & 2.85***                               & 2.46***                     & 2.05***                   & 0.99***                    & 1.5***                             & 3.6***                      & 2.79***                            & 2.04***                   & 2.30                        \\
Adjusted beta        & 1.31                     & 2.50                                    & 2.23                                  & 1.97                        & 1.70                      & 0.99                       & 1.33                               & 2.73                        & 2.19                               & 1.70                      & 1.87                        \\
Avg. mkt. cap (\$bn) & 13.23                    & 7.06                                    & 2.42                                  & 10.95                       & 4.41                      & 0.49                       & 12.85                              & 11.90                       & 14.54                              & 1.25                      & 7.91                        \\ \bottomrule
\end{tabular}}
\smallskip
\caption*{\textit{Notes}: Standard errors in parentheses: *** p$<$0.01, ** p$<$0.05, * p$<$0.1 .}
\end{center}
\end{table}

\begin{figure}[H]
\begin{center}
\caption{Stata script used to run automatic regressions in the crisis regime analysis study.}
\label{Stata_code}
\includegraphics[scale=0.6]{STATA_CODE.png}
\end{center}
\end{figure}

Proof of used GEOMEAN() log transformation:
\begin{equation} \label{geomean}
\begin{split}
GEOMEAN & = (X1*...*Xn)^{(1/n)} \notag \\
ln(GEOMEAN) & = ln((X1*...*Xn)^{(1/n)}) \notag \\ 
ln(GEOMEAN) & = (1/n) * ln(X1*...*Xn) \notag \\ 
ln(GEOMEAN) & = (1/n) * (ln(X1)+...+ln(Xn)) \notag \\ 
ln(GEOMEAN) & = average(ln(X1)+...+ln(Xn)) \notag \\ 
GEOMEAN & = exp(average(ln(X1)+...+ln(Xn))) \notag
\end{split}
\end{equation}

% Bibliography
\newpage
\bibliographystyle{apacite}
\bibliography{Thesis_bibliography}

\end{document}